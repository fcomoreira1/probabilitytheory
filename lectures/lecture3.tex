%! TEX root = ../main.tex
\documentclass[../main.tex]{subfiles}

\usepackage[font, sexy]{moreira}
\usepackage{marginnote}
\reversemarginpar

\author{Francisco Moreira Machado}

\title{Lecture 3}

\begin{document}
We sat that \textbf{Dynkin system} stable by finite intersecions is a $\sigma-$field.

As for $\sigma-$fields, one can show that any intersections of Dynkin systems is a Dynkin
system. \marginnote{Exercise $\rightarrow$}[-0.5cm] This allows us to define

\begin{definition}
    If $\mathcal{C} \subset \mathcal{P}(\Omega)$ is a class of subsets of $\Omega$, we set
    $$\lambda(\mathcal{C}) = \bigcap_{\substack{\mathcal{A} \text{ Dynkin Sys} \\ \mathcal{C}
    \subset \mathcal{A}}}$$
    It is called the \underline{Dynkin system generated by $\mathcal{C}$}.
\end{definition}

\begin{theorem*}
  [Dynkin Lemma]
  Let $\Omega$ be a set. Let $\mathcal{C} \subset \mathcal{P}(\Omega)$ be a class of subsets of
  $\Omega$. Assume that $\mathcal{C}$ is stable by finite intersections then
  \[
  \lambda(\mathcal{C}) = \sigma(\mathcal{C})
  .\] 
  In words, the Dynkin system generated by $\mathcal{C}$ is equal to the $\sigma-$field
  generated by $\mathcal{C}$.
\end{theorem*}
\begin{proof}
  By double inclusion.
  \vspace{0.5em}

  First, since $\sigma(\mathcal{C})$ is a Dynkin system, it must hold that
  $\lambda(\mathcal{C}) \subset \sigma(\mathcal{C})$.
  \vspace{0.5em}

  To show that $\sigma(\mathcal{C}) \subset \lambda(\mathcal{C})$ we show that
  $\lambda(\mathcal{C})$ is stable under finite intersections. Indeed, then it would hold that
  $\lambda(\mathcal{C})$ is a $\sigma-$field, but $\sigma(\mathcal{C})$ is the smallest one
  containing all others, which would finish the proof.

  \vspace{0.5em}

  \underline{Goal:} $\forall A \in \lambda(\mathcal{C}), \forall B \in \lambda(\mathcal{C}), A
  \cap B \in \lambda(\mathcal{C})$.
  \vspace{0.3em}

  \underline{First:} Fix $A \in \mathcal{C}$. We show that $\forall B \in \lambda(\mathcal{C})$
  it holds that $A \cap B \in \lambda(\mathcal{C})$.
  
  \underline{Idea:} Define $\lambda_{A} = \{ B \subset \Omega \colon A \cap B \in \lambda(\mathcal{C})\} $

  \underline{Goal:} $\lambda(\mathcal{C}) \subset \lambda_{A}$.
  We show that $\lambda_{A}$ is a Dynkin system containing $\mathcal{C}$, which will imply the
  desired goal.
  \begin{itemize}
    \item \underline{$\mathcal{C} \in \lambda_{A}$:} If $B \in \mathcal{C}$, we have $A \cap B
      \in \lambda(\mathcal{C})$ due to stability under finite intersection.
    \item \underline{Dynkin system}
      \begin{itemize}
        \item $\Omega \in \lambda_A$ as $A \cap \Omega = A \in \mathcal{C} =
          \lambda(\mathcal{C})$
        \item Take $B \in \lambda_{A}$, then $B^c \in \lambda_{A}$ iff $A \cap B^c = \Omega \setminus \left( (A \cap B)
          \cup A^c \right) $. Moreover, $A \in \mathcal{C}$, so $A^c \in \lambda(\mathcal{C})$
          and $A \cap B \in \lambda(\mathcal{C})$ and they are disjoint sets, hence their union
          must be part of the Dynkin system, after which we conclude by stability under
          complementation.
        \item Take $(B_n)_{n \geq 1}$ pairwise disjoint sequence in $\lambda_A$. Then 
          $\left( \bigcup_{n \geq 1} B_n \right) \cap A = \bigcup_{n \geq 1} B_n \cap A$, but
          the elements of this union are pairwise disjoint in $\lambda(\mathcal{C})$. Hence
          their union must be in $\lambda(\mathcal{C})$ because it is a Dynkin system.
      \end{itemize}
  \end{itemize}
  We then conclude $\lambda(\mathcal{C}) \subset \lambda_{A}$ and so $\forall A \in
  \mathcal{C}$, $\forall B \in \lambda(\mathcal{C})$, $A \cap B \in \lambda(\mathcal{C})$.

  \vspace{0.3em}
  \underline{Second:} Now we fix $A \in \lambda(\mathcal{C})$ and check that
  $\lambda_{A}$ and check that $\lambda_A$ is a Dynkin system containing $\mathcal{C}$. Then
  $\lambda(\mathcal{C}) \subset \lambda_{A}$ and we get $\forall A \in \lambda(\mathcal{C})$,
  $\forall B \in \lambda(\mathcal{C}), A \cap B \in \lambda(\mathcal{C})$ \marginnote{Exercise
  $\rightarrow$}[-0.7cm]
\end{proof}

\noindent In life, Dynkin lemma is often used as follows: 

If $\mathcal{D}$ is a Dynkin system containing a collection $\mathcal{C}$, stable by finite
intersection, then $\sigma(\mathcal{C}) \subset \mathcal{D}$. (Notice that if $\mathcal{D}$ is
a $\sigma-$field, $\mathcal{C} \subset \mathcal{D} \implies \sigma(\mathcal{C}) \subset
\mathcal{D}$).
Indeed, by the Dynkin Lemma, $\sigma(\mathcal{C}) = \lambda(\mathcal{C}) \subset
\lambda(\mathcal{D})$. This justifies the following definition:

\begin{definition}
    Let $(\Omega, \mathcal{A})$ be a measurable space and $\mathcal{C} \subset \mathcal{A}$ a
    collection of measurable sets. We say that $\mathcal{C}$ is a \underline{$\pi-$system} if
    it is stable by finite intersections.

    We say that $\mathcal{C}$ is a \underline{generating $\pi-$system} if $\sigma(\mathcal{C})
    = \mathcal{A}$. 
\end{definition}

\begin{example}
  $\{ (-\infty, a) \colon a \in \rr \} $ is generating of $\mathbb{B}(\rr)$.
\end{example}
\begin{example}
    For $\Omega = \{ 0,1 \} ^{\nn}$ cylinder sets are generating $\pi-$system of the cylinder
    $\sigma-$field.
\end{example}
\begin{corollary}
   Let $(\Omega, \mathcal{A})$ be a measurable space, $\mathcal{C}$ a generating $\pi-$system.
   \begin{enumerate}
     \item Let $\mu, \nu$ be two \underline{finite} measures on $(\Omega, \mathcal{A})$ such
       that $\mu(\Omega) = \nu(\Omega)$ and $\forall A \in \mathcal{C}, \mu(A) = \nu(A)$, then
       $\mu = \nu$.
    \item More generally, if there exists subsets $E_n \in \mathcal{A}$ such that
      $\mu(E_n) = \nu(E_n) < \infty$  $\forall n
      \geq 1$and $\mu(E_n \cap A) = \nu(E_n \cap A) \;\forall A
      \in \mathcal{C}$ and $\bigcup E_n = \Omega$, then $\mu = \nu$
   \end{enumerate}
\end{corollary}
\begin{example}
  [Application to Lebesgue]
  There is at most one measure $\lambda$ on $(\rr, \mathcal{B}(\rr)$ such that $\lambda([a,b])
  = b - a \forall a < b$. This comes from 2. above with $E_n = [-n, n]$.
\end{example}
Probability measures are thus characterized by their values on a generating $\pi-$system.

\begin{proof}[Corollary]
  We show 1. and leave 2. for exercise. \marginnote{Exercise $\rightarrow$}

  \underline{Goal:} $\mu(A) = \nu(A) \forall A \in \mathcal{A}$.

  To do that, take 
  \[
    \mathcal{G} = \{ A \in \mathcal{A} \colon \mu(A) = \nu(A) \} 
  .\]
  We check (exercise) that $\mathcal{G}$ is a Dynkin system containing $\mathcal{C}$, generating
  $\pi-$system, therefore $\mathcal{A} \subset \mathcal{G}$ hence $\forall A \in \mathcal{A}$,
  $\mu(A) = \nu(A)$.

\end{proof}

\subsection{Independence}

Let $(\Omega, \mathcal{A}, \pp)$ be a probability space. Two events $A, B$ are said to be
independent if $\pp(A \cap B) = \pp(A) \pp(B)$.

\underline{Interpretation:} If $\pp(B) > 0$, this is equivalent to $\pp(A \vert B) =
\frac{\pp(A \cap B)}{\pp(B)} = \pp(A)$, which intuitively means that $B$ does not influence the
likelihood of $A$ hapenning.

\begin{example}
  Throw two dice at random $\Omega = [6]^{2}, \pp(\{ \omega \} ) = 1/36
    \forall \omega \in \Omega$,
    then $A = \{ 6 \} \times [6]$ and $B = [6]\times \{ 6 \} $ are independent.
\end{example}
\begin{example}
  Throw one die $\Omega = [6]$ with even probabilities. Then $A = \{ 1,2 \} $ and $B = \{ 1, 3,
  5\} $ are independent.
\end{example}

\begin{definition}
    Events $A_1, \ldots, A_n$ are mutually independent if for every non-empty subset $\{j_1, j_2,
    \ldots, j_k\}$ of $[n]$ we have 
    \[
    \pp \left( A_{j_1} \cap \ldots A_{j_k} \right) = \pp(A_{j_1}) \ldots \pp(A_{j_k})
    .\] 
\end{definition}

\begin{notation}
  $(A_i)_{i \in [n]}$ are  $\indep$.
\end{notation}
\begin{remark}
  Independence is relative to $\pp$. Moreover in general \underline{pairwise independence} does
  not imply independence.
\end{remark}

\begin{proposition}
  Events $A_1, \ldots, A_n$ are $\indep$ iff $\mathbb{P}(B_1 \cap \ldots, B_n) = \pp(B_1)
  \ldots \pp(A_n)$, where $B_i \in \sigma(\{ A_i \}) = \{ \emptyset, A_i, A_i^c, \Omega \} $.

\end{proposition}

\end{document}
