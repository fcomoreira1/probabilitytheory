%! TEX root = ../main.tex
\documentclass[../main.tex]{subfiles}

\title{Lecture 27}

\begin{document}
\begin{obs}
  If $\forall t \in \rr$ $\pp(X_n \leq t)$ has a limit as $n \to \infty$, this
  does \underline{not} imply $X_n$ converge in distribution to $X$.

  Take for example $X_n = n$, $\pp(X_n \leq t) \to 0$, but $0$ is not a cdf of
  a random variable.
\end{obs}
\begin{obs}
  $X_n$ $\rr-$valued, $\forall t \in \rr$, $\phi_{X_n}(t) = \ee[e^{itX_n}]$ has
  a limit as $n \to \infty$ does not imply $X_n$ converge in distribution

  Take for example $X_n \sim N(0, n^2)$. It is clear that its characteristic
  function converges, however if we take $a < b$, then $\pp(a < X_n < b) =
  \pp(a/n < N(0, 1) < b/n) \to 0$. Indeed, if we argue by contradiction assuming
  it converges in distribution to $X$, we could pick $a < b$ such that $\pp(a <
  X < b) \geq 1/2$.
\end{obs}
\begin{remark}
  [Improved Lévy Theorem]
  Assume $\varphi_{X_n}(t) \underset{n \to \infty}{\longrightarrow} f(t)$ $\forall
  t \in \rr$, then $X_n$ converges in distribution iff $f$ is continuous at $0$.
\end{remark}

\subsection{Gaussian vectorrs and the multidimensional CLT}
\begin{definition}
  A r.v. $X = (X_1, \ldots, X_d) \in \rr^d$ is a \textbf{gaussian vector} if any
  linear combination of its cordinates is a gaussian r.v with the convention
  $N(m, 0) = m$ constant.
\end{definition}

Recall that if $X \sim N(0, \sigma^2)$, $\ee[e^{itx}] = e^{itm - \sigma^2
\frac{t^2}{2}} $.

\begin{example}
    \hfill
    \begin{itemize}
      \item If $X_1, \ldots, X_n \indep$ Gaussian r.v. then $(X_1,\ldots,X_d)$
        is a gaussian vector.
      \item If $X, Y$ are $\indep$ Gaussian, $(X, X+Y)$ is a gaussian vector
    \end{itemize}
\end{example}
{\color{red} Warning!} If $(X_1, \ldots, X_d)$ is a gaussian vector, then $X_1,
\ldots, X_d$ are gaussian, but the converse is false.

Indeed take $X \sim N(0, 1)$ and $\varepsilon \sim \pm 1$ with probability
$1/2$, $\indep X$. We can check that $(X, \varepsilon X)$ is not a gaussian
vector since $\pp(X + \varepsilon X) = 1/2$, so $X + \varepsilon X)$ is not
gaussian.

\begin{definition}
  Let $X = (X_1, \ldots, X_d)$ be a gaussian vector.
  \begin{itemize}
    \item $m_X = (\ee[X_1], \ldots, \ee[X_d])$ is called the mean of $X$.
    \item $K_X = (\ee[X_i X_j] - \ee[X_i]\ee[X_j])_{1 \leq i,j \leq d} \in
      \mathcal{M}_{d \times d}(\rr)$ is the covariance matrix of $X$.
    \item $X$ is \underline{centered} if $m_X = (0, 0, \ldots, 0)$.
  \end{itemize}
\end{definition}
\begin{proposition}
    Let $X$ be a gaussian vector in $\rr^d$. Take $\lambda \in \rr^d$, then
    $\langle \lambda, X \rangle = N(m_{\lambda}, \sigma^2_{\lambda})$  with
    $m_{\lambda} = \langle m_X, \lambda \rangle$, $\sigma^2_{\lambda} = \langle
    \lambda, K_X \lambda \rangle$.
\end{proposition}
\begin{corollary}
    $\forall \lambda \in \rr^d$, $\langle \lambda, K_X \lambda \rangle \geq 0$.
    Thus $K_X$ is a positive semi-definite matrix.
\end{corollary}
\begin{corollary}
  The characteristic function of a gaussian vector $X$ is given by
  $\Phi_X(\lambda) = \ee[ e^{i \langle \lambda, X \rangle }] = \exp(i \langle
  \lambda, m_X \rangle - \frac{1}{2} \langle \lambda, K_X, \lambda \rangle)$ for
  $\lambda \in \rr^d$.
  Indeed this is a straight consequence of $\langle \lambda, X \rangle$ being
  gaussian.
\end{corollary}
Since characteristic functions characterize laws, the law of a gaussian vector
$X$ is characterized by $m_X, K_X$.
\begin{application}
    If $X, Y \indep$ gaussian vectors, $X+Y$ is a gaussian vector with $m_{X +
    Y}= m_X + m_Y$ and $K_{X + Y} = K_X + K_Y$.
\end{application}
\begin{remark}
    One can show that if $K$ is $d \times d$ positive semi-definite and $m \in
    \rr^d$, then there exists a gaussian vector $X$ with mean $m$ and covariance
    matrix $K$.
\end{remark}
\begin{theorem}
  \hfill
  \vspace{-2em}

    \begin{enumerate}
      \item Let $X = (X_1,\ldots,X_d)$ be a gaussian vector in $\rr^d$. Then
        $(X_1, \ldots, X_d)$ are $\indep$ iff $K_X$ is diagonal (i.e. $\forall i
        \neq j \ee[X_i X_j] = \ee[X_i]\ee[X_j]$).
      \item Let $Z = (X_1,\ldots, X_p, Y_1,\ldots, Y_q)$ be a gaussian vector in
        $\rr^{p+q}$ then $(X_1, \ldots, X_p) \indep (Y_1, \ldots, Y_q)$ iff
        $\forall 1 \leq i \leq p, 1 \leq j \leq q$, $\ee[X_i Y_j] = \ee[X_i]
        \ee[Y_j]$.
    \end{enumerate}
\end{theorem}
Take home message: for gaussian vectors independence is equivalent to $0$
covariance.

\begin{theorem}
  [Multidimensional CLT]
  Let $(X^i)_{i \geq 1}$ be iid r.v. in $\rr^d$. Assume $\ee[|X^1|] < \infty$.
  Then 
  \[
    \cfrac{X^1 + \ldots + X^n - n \ee[X^1]}{\sqrt{n}}
    \overset{(d)}{\longrightarrow} N(0, K_{X^1})
  .\] 
\end{theorem}
\begin{sketch}
    Similar to $d = 1$, based on characteristic function and taylor expansion of
    $\varphi_{X^1}$ at $0$.
\end{sketch}

\section{A glimpse of statistical theory}

Outline: 
\begin{enumerate}
  \item Estimators
  \item Confidence interval
\end{enumerate}

So far, we use sequences $(X_i)_{i \geq 1}$ of r.v. with known laws. In
statistical theory, it is different: we observe a sequence of values (which we
often assume to be the realization of an iid sequence of r.v.) called
\textbf{sample} but with unknown law.

\underline{\sffamily Goal:} Use the sample to estimate the unknown law or decide
to accept or reject some hypothesis on it.

\subsection{Estimators}
In practice, it often happens that the unkown law belongs to a certain family of
probability measures depending on a parameter $\theta$.

For example: a company would like to sell a product and the goal is to estimate
the proportion $\theta \in [0, 1]$ of people susceptible of buying the product.

\begin{definition}
    A statistical model is a space $\Omega$ with a $\sigma-$field $\mathcal{F}$
    and a family $(P_{\theta})_{\theta \in \Theta}$ of probability measures on
    it, where $\Theta$ is the space of parameters.
\end{definition}

\begin{example}
  \hfill
  \begin{itemize}
    \item $\Theta = [0, 1]$ and $P_{\theta}$ is the law of $Ber(\theta)$.
    \item $\Theta = (0, \infty)$ and $P_{\theta}$ is the law of $Exp(\theta)$.
    \item $\Theta = \rr \times \rr_+$ and $P_{(m, \sigma^2)}$ is the law of
      $N(m, \sigma^2)$.
  \end{itemize}
\end{example}

\begin{definition}
    A sample of size $n$ of a probability measure $P$ is a sequence $X_1,
    \ldots, X_n$ of r.v. $\indep$ with law $P$.

    An \textbf{estimator} is a function $d$ with values in $\Theta$ which
    depends on the sample, i.e. of the form $d(X_1, \ldots, X_n)$. It is
    \textbf{unbiased} if $\forall \theta \in \Theta$, $\ee_{\theta}[d(X_1,
    \ldots, X_n)] = \theta$.
    (when $\Theta \subset \rr^+$, $\ee_\theta$ denotes the expectation with
    respect to $P_\theta$).

    It is strongly consistent if for $\theta \in \Theta$, under $P_\theta$,
    $d(X_1, \ldots, X_n) \overset{a.s.}{\longrightarrow} \theta$.
\end{definition}
    
\end{document}
