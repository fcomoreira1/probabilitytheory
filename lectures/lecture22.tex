%! TEX root = ../main.tex
\documentclass[../main.tex]{subfiles}

\title{Lecture 22}

\begin{document}
\begin{proof}[\boxed{1.}]
  THe idea is to introduce the stopping time $T = \inf \{ k \geq 0 \colon M_k
  \geq a \} $, with $\inf \emptyset = \infty$ and observe that
  $\pp(\max_{0 \leq k \leq k} M_k \geq a) = \pp(T \leq n)$.

  Now, let us expand $a \pp(T \leq n) = a \sum_{k=0}^n  \pp(T = k) =
  \sum_{k=0}^n \ee[a \ind_{a = k}]$, but $a \ind_{a = k} \leq M_k \ind_{a =
  k}$, thus $\sum_{k=0}^n \ee[a \ind_{T = k}] \leq \sum_{k=0}^n \ee[M_k
  \ind_{T = k}]$.

  Moreover, $(M_n)$ is a submartingale, so $M_k \leq \ee[M_n | \mathcal{F}_k]$
  and thus $\sum_{k=0}^n \ee[M_k \ind_{T = k}] \leq \sum_{k=0}^n \ee[\ee[M_n |
  \mathcal{F}_k] \ind_{T = k}] = \sum_{k=0}^n \ee[M_n \ind_{T = k}] = \ee[M_n
  \ind_{T \leq n}]$ which is what we wanted.
\end{proof}

\begin{theorem}
  [Doob $L^p$ inequalities, $p > 1$]
  Fix $p > 1$.

  \begin{enumerate}
    \item Let $(M_n)$ be a positive submartingale, then $\forall n \geq 0$
      \[
        \ee[(\max_{0 \leq k \leq n} M_k)^p] \leq \left( \frac{p}{p-1}
        \right)^p \ee[M_n^p] )
      .\] 
    \item Let $(M_n)$ be a martingale. Write $M_n^* = \max_{0 \leq k \leq n}
      |M_k|$ then
      \[
        \ee[(M_n^*)^p] \leq \left( \frac{p}{p-1}
        \right)^p \ee[|M_n|^p] )
      .\] 
  \end{enumerate}
\end{theorem}
\begin{remark}
  Again, \boxed{2.} follows immediately from \boxed{1.} as if $(M_n)$ is
  martingale, then $(|M_n|)$ is a submartingale.
\end{remark}

Before proving the theorem, we prove some useful results:

\begin{lemma}
  [Holder's inequality]
  Let $q > 1$ be such that $1/p + 1/q = 1$. Let $X, Y$ be $\rr-$valued r.v.
  with $X \in L^p$ and $Y \in L^q$, then
  \[
    \ee[|XY|] \leq \ee[|X|^p]^{1/p} \ee[|Y|^q]^{1/q} 
  .\] 
\end{lemma}
\begin{proof}
  First step: (Young's Inequality) for $a, b \geq 0$, it holds that $ab \leq a^p/p + b^q/q$.

    Second step: We may assume that the expectations on the RHS are positive,
    as otherwise either $X$ or $Y$ would be almost surely $0$, and thus the
    inquality is trivially true. Moreover, we may divide $X$ by
    $\ee[|X|^p]^{1/p}$ and similarly for $Y$ and obtain $\ee[|X|^p] = 1$,
    $\ee[|Y|^q] = 1$.

    Now we apply Young's inequality and conclude $|XY| \leq |X|^p/p +
    |Y|^q/q$, so taking expectations $\ee[|XY|] \leq 1/p + 1/q = 1$.
\end{proof}

\begin{lemma}
  [Moment-tail]
  Let $X \geq 0$ be a random variable, then $\forall p > 0$
  \[
    \ee[X^p] = p \int_{0}^\infty x^{p-1} \pp(X \geq x) dx
  .\] 
\end{lemma}
\begin{proof}
    Using Fubini-Tonelli we get
    \begin{align*}
      p \int_{0}^\infty x^{p-1} \pp(X \geq x) dx 
      &= p \int_{0}^\infty x^{p-1} \ee[\ind_{X \geq x}] dx \\
      &= \ee [\int_{0}^{\infty} px^{p-1} \ind_{X \geq x} dx] \\
      &= \ee[\int_0^\infty p x^{p-1}dx] = \ee[X^p]
    \end{align*}
\end{proof}
\begin{proof}
  [Doob's $L^p$ Inequality]
  If $\ee[M_n^p] = \infty$, then there is nothing to prove. Assume $M_n \in
  L^p$. We further check that for $0 \leq k \leq n$ $M_k \in L^p$ as well: 
  write $\ee[M_k^p] \leq \ee[(\ee[M_n | \mathcal{F}_k])^p] \leq \ee[\ee[M_n^p
  | \mathcal{F}_k]] = \ee[M_n^p] < \infty$ because $(M_n)$ is
  a submartingale and applying conditional Jensen.

  Now write $M_n^* = \max_{0 \leq k \leq n} M_k$. We check $M_n^* \in L^p$ as
  $\ee[(M_n^*)^p] \leq \sum_{k=0}^n \ee[M_k^p] < \infty$.

  Using the tail-moment lemma and Doob's maximal inequality
  \begin{align*}
    \ee[(M_n^*)^p] &= p \int_{0}^\infty a^{p-2} a\pp(M_n^* \geq a) da
    \tag*{Tail-Moment}\\
    &\leq p \int_{0}^\infty a^{p-2} \ee[M_n \ind_{M_n^* \geq a}] da
    \tag*{Doob's Max Ineq}\\
    &= \ee[\int_{0}^\infty pa^{p-2} M_n \ind_{M_n^* \geq a} da] \tag*{Fubini}\\
    &= \ee[p M_n \int_{0}^{M_n^*} a^{p-2}da] \\
    &= p \ee[M_n \frac{(M_n^{*})^{p-1}}{p-1} ] \\
    &\leq \frac{p}{p-1} \ee[M_n^p]^{1/p} \ee[(M_n^*)^p]^{p-1/p} \tag*{By
    Holder} \\
  \end{align*}
  from which the conclusion holds directly.
\end{proof}

\subsection{Martingales bounded in $L^p$}
Recall that for $p>1$, $X$ real-valued r.v. 
\[
  \ee[|X|] \leq \ee[|X|^p]^{1/p}
\] 
So $(X_n)$ bounded in $L^p$ implies it is bounded in $L^1$ and $X_n
\overset{L^p}{\longrightarrow} X \implies X_n \overset{L^1}{\longrightarrow}X$

\begin{theorem}
  [$L^p$ Martingales]
  Let $(M_n)$ be a martingale bounded in $L^p$, $p > 1 (\sup_{n} \ee[|M_n|^p]
  < \infty$. Then
  \begin{enumerate}
    \item $M_n$ converges a.s. and in $L^p$ to a random variable $M_{\infty}$
      with $\ee[|M_{\infty}|^p] = \sum_{n \geq 0}\ee[|M_n|^p]$
    \item Setting $M_{\infty}^* = \sum_{n \geq 0} |M_n|$, we have
      \[
        \ee[(M_{\infty}^*)^p] \leq \left(\frac{p}{p-1}\right)^p
        \ee[|M_{\infty}|^p] 
      .\] 
  \end{enumerate}
\end{theorem}
\begin{proof}
  For \boxed{1.} $(M_n)$ is bounded in $L^p$, so it is bounded in $L^1$ and in
  particular, it converges a.s to some r.v. $M_\infty$. 

  To show that this
  convergence also holds in $L^p$, we use Doob's $L^p$ inequality:
  $\ee[(M_{\infty}^*)^p] \leq (p/(p-1))^p \sup_{k \geq 0} \ee[|M_k|^p]$, but 
  $M_n^*$ converges increasingly to $M_{\infty}^*$, so by monotone convergence
  $\ee[(M_n^{*})^p] \underset{n \to \infty}{\longrightarrow}
  \ee[(M_{\infty}^*)^p]$ and we conclude that $\ee[(M_{\infty}^*)^p] \leq
  (p/(p-1))^p \sup_{k \geq 0} \ee[|M_k|^p]$ thus $M_{\infty}^* \in L^p$ and so
   $M_{\infty} \in L^p$. 

   Now, notice $|M_n - M_{\infty}|^p \overset{a.s.}{\longrightarrow} 0$ and
   $|M_n - M_{\infty}|^p \leq (|M_n| + |M_{\infty}|)^p \leq 2^p (|M_n|^p +
   |M_{\infty}|^p) \leq 2^p (|M_{\infty}^*|^p +
   |M_{\infty}|^p)$, so it is bounded and we may apply dominated convergence
   to conclude convergence in $L^p$.
  Now, since $M_n \overset{L^p}{\longrightarrow} M_{\infty}$ we
   have $\ee[|M_n|^p] \to \ee[|M_{\infty}|^p]$. Since $(|M_n|^p)$ is a
   submartingale, the sequence $(\ee[|M_n|^p|])$ is nondecreasing, so the
   limit and the supremum coincide
\end{proof}
\begin{remark}
  If $(M_n)$ converges in $L^p$, then it is bounded in $L^p$ (true for any
  sequence of random variables).
\end{remark}

\section{Convergence in distribution of random variables}
In a.s., $\pp$, $L^p$ convergence of random variables $X_n \to X$, the
quantity "$X_n(\omega) - X(\omega)$" was involved, it says something about the
joint realization of $X_n$ and the limit $X$.

Here we define a notion of convergence for the \underline{laws} of random
variables.


\subsection{Definition and first properties}

We work with $\rr^d-$valued random variables (but most that follows can be
extended to general metric spaces).

\begin{notation}
  [Set of bounded continuous functions]
  $\mathcal{C}_b(\rr^d) = \{ f: \rr^d \to \rr \text{ cotinuous and bounded}\}
  $. For $f \in \mathcal{C}_b(\rr^d)$, we write $\| f \|_{\infty} = \sup_{x
  \in \rr^d} |f(x)|$. Here $|\cdot|$ is any norm in $\rr^d$.
\end{notation}

\begin{definition}
  A sequence $(\mu_n)$ of probability measures on $\rr^d$ is said to converge
  \underline{weakly} to a probability measure $\mu$ on $\rr^d$ if $$\forall f
  \in \mathcal{C}_b(\rr^d), \int_{\rr^d} f(x)\mu_n(dx) \underset{n \to \infty}{\longrightarrow} \int_{\rr^d}
  f(x)\mu(dx) $$
  ($f$ is called a test function).

  Moreover, a sequence $(X_n)$ of $\rr^d-$valued r.v. is said to converge in
  distribution or converge in law to a $\rr^d-$valued r.v. if $\pp_{X_n} \to
  \pp_X$ weakly, that is
  \[
    \forall f \in \mathcal{C}_b(\rr^d), \ee[f(X_n)] \underset{n \to
    \infty}{\longrightarrow}\ee[f(X)]
  .\] 
\end{definition}
\begin{remark}
    When we say $X_n$ converges in distribution to $X$, there is an abuse of
    notation. The limiting random variable is not uniquely defined, only its
    LAW is.

    For this, we sometimes say that "$X_n$ converges in distribution to $\mu$"
    a probability measure.

    Finally, the random variables $(X_n), X$ are not necessarily defined on
    the same space.
\end{remark}

\end{document}
