%! TEX root = ../main.tex
\documentclass[../main.tex]{subfiles}

\title{Lecture 23}
\begin{document}
\begin{theorem*}
  [Probabilistic Formulation]
  Let $X_n, X$ be r.v. in $\rr^d$. The following are equivalent
  \begin{enumerate}
    \item $X_n \overset{(d)}{\longrightarrow} X$
    \item $\forall f : \rr^d \to \rr$ lipschitz bounded $\ee[f(X_n)] \overset{}{\longrightarrow} \ee[f(X)]$
    \item $\forall F \subset \rr^d$ closed, $\limsup_{n \to \infty} \pp(X_n \in
      f) \leq \pp(X \in f)$
    \item $\forall O \subset \rr^d$ open, $\liminf_{n \to \infty} \pp(X_n \in
      O) \geq \pp(X \in O)$
    \item $\forall A \subset \rr^d$ with $\pp(X \in \partial A) = 0$, $\pp(X_n
      \in A) \to \pp(X_A)$.
    \item   $\forall f: \rr^d \to \rr$ measurable bounded, a.s. continuous at 
      $X$, $\ee[f(X_n)] \longrightarrow \ee[f(X)]$.
  \end{enumerate}
\end{theorem*}
\begin{corollary}[Extended Continuous Mapping]
  If $X_n \overset{(d)}{\longrightarrow} X$, $F : \rr^d \to \rr^n$ is almost
  surely continuous at $X$, then $F(X_n) \overset{(d)}{\longrightarrow} F(X)$.
\end{corollary}
\begin{proof}
    This comes from the fact that if $f : \rr^n \to \rr$ is continuous bounded,
    then $f \circ F \colon \rr^d \to \rr$ is bounded, almost surely continous at
    X and the result follows from \boxed{6.}
\end{proof}

\begin{example}
  If $X_n$ is $\rr$ valued and $X_n \overset{(d)}{\longrightarrow} X$ with $X
  \neq 0$ a.s. then $1/X_n \overset{(d)}{\longrightarrow} 1/X$
\end{example}

\vspace{0.4em}
\noindent
\underline{\sffamily Connection with CDF's in $\rr$}

\vspace{0.4em}
\noindent
If $X$ is a $\rr-$valued r.v., $F_X(t) = \pp(X \leq t)$ for $t \in \rr$ is its
CDF.

\begin{itemize}
  \item $F_X$ is continuous at $x$ iff $\pp(X=x) = 0$
  \item $F_X$ has at most a countable number of discontinuity points
\end{itemize}

\begin{theorem}
  Let $X_n, X$ be a $\rr-$valued r.v. then $X_n \overset{(d)}{\longrightarrow}
  X$ iff $\pp(X_n \leq t) \to \pp(X \leq t)$ for every $t \in \rr$ 
  that is a continuity point of $F_x$.
\end{theorem}
\begin{example}
  $X_n = 1/n$, $X_n \overset{(d)}{\longrightarrow} 0$.
\end{example}

\begin{proof}
  $\boxed{\implies}$ Let $t \in \rr$ be a continuity point of $F_X$, so $\pp(X =
  t)  = 0$. Take $A = (-\infty, t]$ in \boxed{5.} of Portemanteau, $\partial A =
  \{ t \} $ so $\pp(X \in \partial A) =\pp(X = t) = 0$. Thus $F_{X_n}(t) =
  \pp(X_n \in A) \to \pp(X \in A) = F_X(t)$.

  \vspace{0.4em}
  \noindent
  \boxed{\impliedby} We show \boxed{4.} in Portemanteau, i.e. $\forall O \subset
  \rr$ open, $\liminf_{n \to \infty} \pp(X_n \in O) \geq \pp(X \in O) (\star)$

  We show first that $\forall a, b \in \rr$, $\limsup_{n \to \infty} \pp(X_n
  \leq a) \leq \pp(X \leq a)$ and $\liminf_{n \to \infty} \pp(X_n < b) \geq \pp(X <
  b) $, putting this together $(\star)$ will hold for all open intervals.

  Since $F_X$ has at most countable number of discontinuity points, its
  continuity points are dense in $\rr$, so we can choose $t > a$ with $F_X$
  continuous at $t$. 

  Then $\limsup_{n \to \infty} \pp(X_n \leq a) = \limsup_{n \to \infty} \pp(X_n
  \leq t) = \pp(X \leq t)$ by assumption. Now take $t$ converge decreasingly to
  $b$ and this together with the right continuity of the cdf implies $\pp(X \leq
  t) \to \pp(X \leq a)$. Finally, to prove the liminf result is very similar \marginnote{Ex.
  $\rightarrow$}

  \vspace{0.4em}
  \noindent
  Now we go back to taking $O \subset \rr$ open. We know that we can write $O =
  \bigcup_{i \in I} (a_i, b_i)$ with $I$ being at most countable and $(a_i,
  b_i)$ being pairwise disjoint open intervals. In particular
  \begin{align*}
    \pp(X \in O)  &= \pp(X \in \bigcup_{i \in I} (a_i, b_i)) = \sum_{i \in I}
    \pp(X \in (a_i, b_i)) \\
    &\leq \sum_{i \in I} \liminf_{n \to \infty} \pp(X_n \in (a_i, b_i )) \\ 
    &\leq \liminf_{n \to \infty} \sum_{i \in I} \pp(X_n \in (a_i, b_i))
    \tag*{ by Fatou} \\
    &= \liminf_{n \to \infty} \pp(X_n \in O)
  \end{align*}
\end{proof}

\begin{corollary}
  $X_n \overset{(d)}{\longrightarrow} X$ with density $p$ iff $\forall t \in
  \rr, \pp(X_n \leq t) \to \pp(X \leq t)$ iff $\forall t \in \rr, \pp(X_n < t)
  \to \pp(X < t) = \pp(X \leq t)$ iff $\forall a < b$ $\pp(a \leq X_n \leq b)
  \to \int_{a}^b p(t) dt$.
\end{corollary}
\begin{application}
  Fix $\lambda > 0$, and take $X_n \sim Geo(\frac{\lambda}{n} )$, then $X_n / n
  \overset{(d)}{\longrightarrow} Exp(\lambda)$. 
\end{application}

\begin{proposition}
    Let $X_n$ be $\rr^d$ valued and $a \in \rr^d$ a constant, then $X_n
    \overset{(d)}{\longrightarrow}a$ iff $X_n \overset{\pp}{\longrightarrow}a$
\end{proposition}
\begin{proof}
  $\boxed{\impliedby}$ We have already seen that convergence in probability
  implies convergence in distribution 

  $\boxed{\implies}$ We show $\forall \varepsilon > 0$, $\pp(|X_n - a| \geq
  \varepsilon) \to 0$. Take $B(x, \varepsilon)$ to be the open ball of radius
  $\varepsilon$ around $x$, in particular, $\pp(|X_n - a| \geq \varepsilon) =
  \pp(X_n \in B(a, \varepsilon)^c)$. Then by Portemanteau for closed sets
  \[
    \limsup_{n \to \infty} \pp(|X_n - a| \geq \varepsilon) \leq \pp(a \in B(a,
    \varepsilon)^c) = 0
  .\] 
\end{proof}
\begin{theorem}
  [Slutsky's Theorem]
  Let $X_n, X, Y_n$ be $\rr^d-$valued random variable, $a \in \rr^d$ constant.
  Assume $X_n \overset{(d)}{\longrightarrow} X$, $Y_n
  \overset{\pp}{\longrightarrow} a$, then $(X_n, Y_n)
  \overset{(d)}{\longrightarrow} (X, a)$.
\end{theorem}
\begin{application}
  If $a = 0$, then $X_n + Y_n \overset{(d)}{\longrightarrow} X$. Indeed, $(X_n,
  Y_n) \overset{(d)}{\longrightarrow} (X, 0)$, thus by continuous mapping
  $f(X_n, Y_n) \overset{(d)}{\longrightarrow} f(X, 0)$ with $f(x, y) = x+y$

  Moreover, if $a \neq 0$, then $X_n / Y_n \overset{(d)}{\longrightarrow} X /
  a$, which we can prove by extended continuous mapping with $f(x, y) = x / y$
  if $y \neq 0$ and $0$ otherwise. One can check that $f$ is almost surely
  continuus at $(X, a)$.
\end{application}

\underline{\sffamily Take home message:} in a cv in $(d)$ one can replace a
random varirable by its limiting values when it converges in probability without
changing the limit.

{\color{red} Warning!} In general, $X_n \overset{(d)}{\longrightarrow} X, Y_n
\overset{(d)}{\longrightarrow} Y$ does not imply $(X_n, Y_n)
\overset{(d)}{\longrightarrow} (X, Y)$. Indeed take $X$ with $\pp(X = 1) = \pp(X
= -1/2) = 1/2$ and $X_n = X$, $Y_n = -X$, then it will not hold.

We will prove later that the implication works under assumption of $\indep$.

\begin{lemma}
  Let $X_n, X, Y_n$ be $\rr^d-$valued. Assume $X_n
  \overset{(d)}{\longrightarrow} X$ and $|X_n - Y_n|
  \overset{\pp}{\longrightarrow} 0$, then $Y_n \overset{(d)}{\longrightarrow} X$
\end{lemma}
\begin{proof}
  We show that $\forall F$ closed, $\limsup \pp(Y_n \in F) \leq \pp(X \in F)$.
  Define for $p \geq 1$ $F^{(1/p)} = \{ x \in \rr^d \colon d(x, F) \leq
  \frac{1}{p}  \} $ called the $1/p-$closed enlargement of $F$.
  \begin{align*}
    \pp(Y_n \in F) &= \pp(Y_n \in F, |X_n - Y_n| \leq \frac{1}{p} ) + 
    \pp(Y_n \in F, |X_n - Y_n| > \frac{1}{p} )  \\
    &= \pp(X_n \in F^{(1/p)}) + \pp(|X_n - Y_n| > \frac{1}{p}).
  \end{align*}
  So $\limsup \pp(Y_n \in F) \leq \pp(X \in F^{(1/p)}) + 0$. 

  Now take $p \to \infty$ since $F^{(1/p)}$ is decreasing and $\bigcap_{p \geq
  1} F^{(1/p)} = F$ as $F$ is closed, we get $\pp(X \in F^{(1/p)}) \underset{p
  \to \infty}{\longrightarrow} \pp(X \in F)$.
\end{proof}
\begin{proof}
  [Slutsky's Theorem]
  By continuous mapping, we have $(X_n, a) \overset{(d)}{\longrightarrow} (X,
  a)$. 

  Now equp $\rr^2$ with the $L^1$ norm and observe that $|(X_n, a) - (X_n, Y_n)|
  = |Y_n- a| \overset{\pp}{\longrightarrow} 0$ by assumption, thus by the lemma
  $(X_n, Y_n) \overset{(d)}{\longrightarrow} (X, a)$.
\end{proof}

\subsection{Restricting Test Functions}
Let $\mathcal{C}_c(\rr^d) = \{ f \colon \rr^d \to \rr \text{ continuous with
compact support} \} $.

\begin{theorem}
    Take, $\mu_n, \mu$ prob measures on $\rr^d$. Then $\mu_n \to \mu$ weakly iff
    $\forall f \in \mathcal{C}_c(\rr^d)$, $$\int_{\rr^d} f(x) \mu_n(dx) \to
    \int_{\rr^d}f(x)\mu(dx)$$
\end{theorem}
{\color{red} Warning!} this result is specific to $\rr^d$ and is not true for
general metric spaces.
\begin{proof}
  $\boxed{\implies}$ is clear because $\mathcal{C}_c(\rr^d) \subset
  \mathcal{C}_b(\rr^d)$.

  $\boxed{\impliedby}$ Take $f \in \mathcal{C}_b(\rr^d)$, let us use a
  truncation argument. 

  Take $R > 1$ and define $g_R(x) = 1$ if $|x|< R$ and
  $\max(R+1 - |x|, 0)$ if $|x| \geq R$. 
  Notice $f g_R \in \mathcal{C}_c (\rr^d)
  \forall r \geq 1$.

  For $R > 0$ fixed, 
  \begin{align*}
    \left| \int f(x)\mu_n(dx) - \int f(x) \mu(dx) \right| &\leq \int |f(x) -
    f(x) g_R(x) \mu_n(dx)| \\ &+ \left| \int f(x)g_R(x)\mu_n(dx) - \int f(x) g_R(x)
    \mu(dx)\right| \\ &+ \int |f(x) - f(x)g_R(x)|\mu(dx)  \\
  \end{align*} 

  Hence, taking limsup on both sides and using tkat $g_R, fg_R \in
  \mathcal{C}_c(\rr^d)$
  \begin{align*}
    \limsup_{n \to \infty}
    \left| \int f(x)\mu_n(dx) - \int f(x) \mu(dx) \right| &\leq
    \limsup_{n \to \infty }\|f\|_{\infty}( 1 - \int g_R(x)\mu_n(dx)) + 0 \\ &+ \|f\|_{\infty} (1 - \int
    g_R(x)\mu(dx)) \\
    &= 2 \|f\|_{\infty} (1 - \int
    g_R(x)\mu(dx))
  \end{align*}
  But $\int g_R(x) \mu(dx) \underset{R \to \infty}{\longrightarrow} 1$ by
  dominated convergence, finishing the proof
\end{proof} 
\end{document}
