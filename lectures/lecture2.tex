%! TEX root = ../main.tex
\documentclass{scrartcl}

\usepackage[font, sexy]{moreira}
\usepackage{marginnote}
\reversemarginpar

\author{Francisco Moreira Machado}

\title{Lecture 2}

\begin{document}
  % \maketitle
  \begin{example}
    [$\sigma$-field]
    Take $\Omega = \{ 0, 1 \}^{\{ 1,2, \ldots \} } $  $=\{(x_n)_{n \geq 1} \colon x_i \in \{
    0,1 \} \,\forall i \geq 1 \} $
    which can model the outcomes of throwing infinitely manay times a coin. 
  \end{example}
  \begin{definition}
    [Cylinder Set]
    We say that  asubset of $\Omega$ is a \textbf{cylinder set} (or, in short, a cylinder)  if
    it is of the form 
    \[
      \mathcal{C}_{a_1, \ldots, a_k} = \{ (x_n)_{n \geq 1} \colon x_1 = a_1, \ldots, x_k = a_k
      \}, \text{ with } a_i \in \{ 0,1 \}
    \] 
    It represents outcomes where the first $k$ results are fixed.
  \end{definition}

  The cylinder $\sigma-$algebra $\mathcal{C}_{cyl}$ is defined to be the 
  $\sigma-field$ generated by the cylinders.

  \begin{example}
      $\{ (1, 1, \ldots) \}  \in \mathcal{C}_{cyl}$ because it is the same set as $\bigcap_{n
      \geq 1} \mathcal{C}_{\substack{ \underbrace{1, \ldots, 1}_{n \text{ times}}}}$
  \end{example}

  \begin{example}
    Take $\Omega = \rr$ and $\mathcal{A} = \sigma(\{ x \}, x \in \rr)$, one can check that
    $\mathcal{A} = \{ A \subset \rr \colon A \text{ or } A^c \text{ is countable}\} $. 
  \end{example}
  {\color{red} Warning} In general elements of generated $\sigma-$fields are not "explicit".

  \begin{definition}
    {Borel $\sigma-$field}
    If $(E, d)$ is a metric space (take $E = \rr$), the \textbf{Borel $\sigma-$field} is 
    $\sigma(\{ U \colon U \subset E, U \text{ open set} \})$. It is denoted by $\mathcal{B}(E)$
    or $\mathcal{B}_E$. It is also the $\sigma-$field generated by closed sets.
  \end{definition}
  \begin{example}
      for $E = \rr$ one can check that
      \begin{align*}
        \mathcal{B}(E) &= \sigma(]a, b[, a < b, a, b \in \rr) \\
                       &= \sigma(]-\infty, b[, b \in \rr) \\
                       &= \sigma(]-\infty, b), b \in \rr) \\
      \end{align*}
      For this, the \underline{key property} is that any open set of $\rr$ is a countable
      disjoint union of open intervals.
  \end{example}

  \begin{definition}[Product $\sigma-$field]
    Let $(E, \mathcal{E})$ and $(F, \mathcal{F})$ be two mesurable spaces. The \textbf{product
    $\sigma-$field} $\mathcal{E} \times \mathcal{F}$ is 
    \[
      \mathcal{E} \times \mathcal{F} = \sigma(A \times B \colon A \in \mathcal{E}, B \in
      \mathcal{F})
    .\] 
    It is the smallest $\sigma-$field on $E \times F$ containing elements $A \times B$ with $A
    \in \mathcal{E}, B \in \mathcal{F}$.
  \end{definition}

  \subsection{Measures}
  \begin{definition}
      A measure on a measurable space $(\Omega, \mathcal{A})$ is a function $\mu
      \colon \mathcal{A} \to \mathbb{R}_+ \cup \{ \infty \} $ with 
      \begin{enumerate}
        \item $\mu(\emptyset) = 0$.
        \item If $(A_n)_{n \geq 1}$ is a (countable) sequence of pairwise disjoint elements of
          $\mathcal{A}$, then $\mu\left(\bigcup_{n \geq 1} A_n\right) = \sum_{n \geq 1} \mu(A_n)$
      \end{enumerate}
  \end{definition}

  When $\mu(\Omega)$ is finite, we say that $\mu$ is a finite measure. Moreover, when
  $\mu(\Omega) = 1$ we say that $\mu$ is a \textbf{probability measure}, we usually write
  $\mathbb{P}, \mathbb{Q}$ instead of $\mu$. Then $(\Omega, \mathcal{A}, \mu)$ is called a
  probability space.

  \begin{proposition}
      Let $\mu$ be a measure on $(\Omega, \mathcal{A})$
      \begin{enumerate}
        \item For $A, B \in \mathcal{A}, $ if $A \subset B$ then $\mu(B \setminus A) + \mu(A) =
        \mu(B)$. If $\mu(A) < \infty$, then $\mu(B \setminus A) = \mu(B) - \mu(A)$.
        \item If $(A_i)_{i \geq 1}$ are measurable and $A_1 \subset A_2 \ldots$ then $\mu
        (\bigcup_{n \geq 1} A_n) = \lim_{n \to \infty} \mu(A_n)$.
        \item If $(A_i)_{i \geq 1}$ are measurable and $A_1 \supset A_2 \ldots$ and $\mu(A_1)
          < \infty$ then $\mu(\bigcap_{n \geq 1} A_n) = \lim_{n \to \infty} \mu(A_n)$.
        \item If $(A_i)_{i \geq 1}$ are measurable, $\mu(\bigcup_{n \geq 1} A_n) \leq \sum_{n
          \geq 1}^\infty \mu(A_n)$.
      \end{enumerate}
  \end{proposition}

  \begin{proof}

    1. Comes from the second property on the definition by taking $A_1 = B \setminus A, A_2 =
    A, A_3 = \emptyset = A_4 \ldots$.
    \vspace{0.5em}

    2. Set $B_1 = A_1$ and $B_{i + 1} = A_{i + 1} \setminus A_i$ for $i \geq 1$, they are
    pariwise disjoint and $B_1 \cup B_2 \ldots B_k = A_k$. Hence $\bigcup_{n \geq 1} A_n =
    \bigcup_{n \geq 1} B_n$ thus $\mu \left( \bigcup_{n \geq 1} A_n \right) = \mu \left(
    \bigcup_{n \geq 1} B_n \right) = \sum_{n \geq 1}^\infty \mu(B_n) = \lim_{n \to \infty}
    \sum_{k = 1}^n \mu(B_k) = \lim_{n \to \infty} \mu \left( \bigcup_{k = 1}^n B_k \right) =
    \lim_{n \to \infty} \mu(A_n) $.
    \vspace{0.5em}

    3. \textbf{Complementation Trick} apply 2. with $(A_i^c)_{i \geq 1}$ \marginnote{Exercise
    $\rightarrow$}
    \vspace{0.5em}

    4. Since $B\setminus A \cap B \subset B$, we have $\mu(A\cup B) = \mu(A) + \mu(B
    \setminus A) \leq \mu(A) + \mu(B)$. Hence by induction $\mu \left( \bigcup_{i =1}^n A_i
    \right) \leq \sum_{i=1}^n \mu(A_i)$. As we apply limits we get by 2. $\mu \left( \bigcup_{n
    \geq 1} A_i \right) \leq \sum_{n \geq 1}A_n $.
  \end{proof}


  \begin{example}
    [The Counting Measure]
    The cardinality on a set $E$ is defined by
    $Card(B)$ and can be  used when $E$ is finite or countable
  \end{example}
  \begin{example}
    [The Dirac Mass]
    is a measure fo the form $\delta_a$ for $a \in \Omega$ defined by 
    $\delta_a(A) = \ind_{a \in A}$.
  \end{example}
  \begin{example}
    [Lebesgue Measure]
    The Lesbegue measure $\lambda$ on $(\rr, \mathcal{B}(\rr)$ satisfies $\lambda([a, b]) = b
    - a$ for $a < b$. 
  \end{example} 

  Observe that any positive linear combination of measures is a measure on $(\Omega,
  \mathcal{A})$.

  \begin{remark}
    Recall $\Omega = \{ 0,1 \} ^{\{1, 2, \ldots\}}$ and $\mathcal{C}_{a_1, \ldots, a_k}$. One
    can show that there does not exist a \textit{probability measure} $\mu$ on $(\Omega,
    \mathcal{P}(\Omega))$ such that $\mu(\mathcal{C}_{a_1, \ldots, a_k}) = 2^{-k}$. This is due
    to the $\mathcal{P}(\Omega)$ being "too large".
    \vspace{0.3em}

    \underline{\textbf{BUT}} there is one on $(\Omega, \mathcal{C}_{cyc})$.
  \end{remark}

  \begin{notation}
    $\mu$ measure on $(\Omega, \mathcal{A})$
    \begin{itemize}
      \item $\mu$ is \underline{$\sigma-$finite} if $\exists (A_n)_{n \geq 1)}$ sequence of $\mathcal{A}$
        such that $\mu(A_n) < \infty$ for all $n \geq 1$ and $\Omega = \bigcup_{n \geq 1} A_n$
      \item $x \in \Omega$ is an \underline{atom} if $\mu(\{ x \} ) > 0$.
    \end{itemize}
  \end{notation}

  If $\mu$ has no atoms, we say that $\mu$ is \underline{non-atomic}. If $\mu$ is a (weighted)
  sum of Dirac masses, we say that $\mu$ is \underline{atomic}.

  \begin{example}
  \begin{itemize}

    \item $\lambda$ (Lebesgue) is atomic
    \item $\delta_3 / 3 + 5 \delta_{\frac{\sqrt{17} - 1}{2}}$ is atomic
    \item $\lambda + \delta_2$ is neither.
  \end{itemize}
  \end{example}

  \subsection{The Dynkin Lemma}
  \begin{definition}
    Let $\mathcal{D} \subset \mathcal{P}(\Omega)$ be a collection of subsets of $\Omega$. We
    say that $\mathcal{D}$ is a \underline{Dynkin system} (or \underline{$\lambda$-system}) if
    \begin{enumerate}
      \item $\Omega \in \mathcal{D}$.
      \item If $A \in \mathcal{D}$, then $A^c \in \mathcal{D}$.
      \item If $(A_n)_{n \geq 1}$ is a countable sequence in $\mathcal{D}$ of \textit{pairwise
        disjoint} sets, then $\bigcup_{n \geq 1} A_i \in \mathcal{D}$.
    \end{enumerate}
  \end{definition}
  In particular, a $\sigma-$field is a \textit{Dynkin system}, but the converse is false on
  $\Omega = \{ 0,1,2,3\} $ take $\mathcal{D} = \{ \emptyset, \Omega, \{ 0,1 \}, \{ 2,3 \} , \{
  0,2\} , \{ 1,3 \}  \} $ and check that it is a Dynkin system but not a $\sigma-$field

  \begin{lemma}
    Assume that $\mathcal{D} \subset \mathcal{P}(\Omega)$ is a Dynkin system. Assume that it is
    stable by finite intersections, then $\mathcal{D}$ is a $\sigma-$field.
  \end{lemma}

  \begin{proof}
      It suffices to prove the last condition of a $\sigma-$algebra. Let $(A_n)_{n \geq 1}$ be
      in $\mathcal{A}$ we show that $\bigcup_{n \geq 1} A_n \subset \mathcal{D}$. Let $B_1 =
      A_1$ and for $j \geq 2$ set $B_j = A_j \setminus (A_1 \cup \ldots A_{j - 1})$. By
      construction $B_1 \cup \ldots \cup B_j = A_1 \cup \ldots \cup A_j$ and the $(B_j)$ are
      disjoint. We show by strong induction that $\forall j \geq 1$, $B_j \in \mathcal{D}$. 

      It is direct for $j = 1$, and now if we assume $B_1, \ldots, B_j \in \mathcal{D}$ then
      \begin{align*}
        B_{j+1} &= A_{j+1} \setminus (A_1 \cup \ldots A_j) \\
        &= A_{j+1} \setminus (B_1 \cup \ldots B_j) \\
        &= A_{j+1} \cap (\Omega \setminus (B_1 \cup \ldots B_j)) \in \mathcal{D}
      \end{align*}
      as $\mathcal{D}$ is closed under interscetion. Moreover, as each $B_j \in \mathcal{D}$,
      we have that their union also does, finishing the proof.
  \end{proof}
\end{document}
