%! TEX root = ../main.tex
\documentclass[../main.tex]{subfiles}

\author{Francisco Moreira Machado}

\title{Lecture 13}

\begin{document}
   \subsection{Different Notions of Convergence} 

   Let $X, X_n$ be random variables in $\rr^k$ (with
   any norm). We have already seen the notion of almost sure convergence:
   \[
     X_n \overset{a.s}{\longrightarrow} X \;\;\text{ if } \pp(\{\omega \in \Omega
     \colon X_n(\omega) \to X(\omega) \} ) = 1
   \] 

   \begin{definition}
     We say that $X_n \to X$ in probability and write $X_n
     \overset{\pp}{\longrightarrow} X$ if $\forall \varepsilon>0$, $\pp(|X_n -
     X| \geq \varepsilon) \underset{n \to \infty}{\longrightarrow} 0$.
     Here the norm $|\cdot|$ is the norm in $\rr^k$.

     \vspace{0.2em}

     \noindent
     If $X_n, X$ are $\rr-$valued, we say that $X_n$ converges to $X$ in $L^p$ if 
     $\ee[|X_n-X|^p] \underset{n\to \infty}{\longrightarrow} 0$.
   \end{definition}

   \begin{remark}
     Almost sure convergence involves the joint law of $(X, X_1, X_2, \ldots)$
     while convergence in probability and $L^p$ only involve the joint law of
     $(X_n, X)$.
   \end{remark}

   \begin{remark}
     By monotonicity, $\varepsilon' > \varepsilon$ then $\pp(|X_n - X| \geq
     \varepsilon') \leq \pp(|X_n - X| \geq \varepsilon)$, so $ X_n
     \overset{\pp}{\longrightarrow} X$ if $\forall \varepsilon>0$ small enough
     the condition holds.
   \end{remark}

   \begin{proposition}
     $X_n \overset{\pp}{\longrightarrow} X$ iff $\ee[\min(|X_n - X|, 1)]
     \to 0$. 
   \end{proposition}
   \begin{proof}
     $\boxed{\Rightarrow}$ Take $\varepsilon>0$ and write $$\ee[min(|X_n - X|,
     1)] = \ee[\min(|X_n - X|, 1) \ind_{|X_n - X| < \varepsilon}] + 
     \ee[\min(|X_n - X|, 1) \ind_{|X_n - X| \geq \varepsilon}]$$
     Moreover, we have $\ee[\min(|X_n - X|, 1) \ind_{|X_n - X| < \varepsilon}]
     \leq \ee[\varepsilon] = \varepsilon$ and $\ee[\min(|X_n - X|, 1)
     \ind_{|X_n - X| \geq \varepsilon}] \leq \ee[\ind_{|X_n - X| \geq
     \varepsilon}] = \pp(|X_n - X| \geq \varepsilon)$ 
     \vspace{0.3em}

     \noindent
     Thus $\limsup_{n \to \infty} \ee[\min(|X_n - X|, 1)] \leq
     \varepsilon$, which holds for all $\varepsilon$, so it must be $0$.

     \vspace{1em}
     \noindent
     $\boxed{\Leftarrow}$ Take $\varepsilon \in [0, 1]$ and observe that $|X_n
     - X| \geq \varepsilon \implies \min(|X_n - X|, 1) \geq \varepsilon$, thus
     $\pp(|X_n - X| \geq \varepsilon) \leq \pp(\min(|X_n - X|, 1) \geq
     \varepsilon) \leq \ee[min(|X_n - X|, 1)] / \varepsilon \underset{n \to
     \infty}{\longrightarrow} 0$, by Markov's inequality.
   \end{proof}

   \begin{proposition}
       If $X_n \overset{a.s}{\longrightarrow} X$ or $X_n
       \overset{L^p}{\longrightarrow} X$ then $X_n
       \overset{\pp}{\longrightarrow} X$
   \end{proposition}
   \begin{proof}
       Assume $X_n \overset{L^p}{\longrightarrow} X$. Fix $\varepsilon>0$ and
       write $\pp(|X_n - X| \geq \varepsilon) = \pp(|X_n - X|^p \geq
       \varepsilon^p) \leq \ee[|X_n - X|^p] / \varepsilon^p \underset{n \to
       \infty}{\longrightarrow}0$ again by Markov's Inequality.

       \vspace{1em}
       \noindent
       Assume that $X_n \overset{a.s.}{\longrightarrow} X$. Now observe that
       $\min(|X_n - X|, 1) \overset{a.s.}{\longrightarrow} 0$ and $0 \leq
       \min(|X_n - X|, 1) \leq 1$, hence by Dominated Convergence we get the
       result.
   \end{proof}
   \begin{remark}
     For $p = 2$ and $X = \ee[X_n]$
     the inequality
       \[
         \pp(|X_n - \ee[X_n]| \geq \varepsilon) \leq \frac{\ee[(X_n -
        \ee[X])^2]}{\varepsilon^2} = \frac{Var(X_n)}{\varepsilon^2}
       \]
       is know as the Bienaymé-Tchebyshev Inequality.
   \end{remark}
   \begin{example}
     Fix $\alpha > 0$ and let $(X_n)_{n \geq 1}$ be $\indep$ r.v. with
     $\pp(X_n=1) = 1/n^\alpha$ and $\pp(X_n = 0) = 1 - 1/n^\alpha$.

     \vspace{1em}

     \noindent
     For this, we can compute $\ee[X_n^p] = 1/n^\alpha \underset{n \to
     \infty}{\longrightarrow} 0$, hence it converges in $L^p$ and probability
     to $0$.
     
     \vspace{0.5em}
     \noindent\underline{What about a.s convergence?}
     \newline
     For $\alpha > 1$, we have that $\sum_{n=1}^\infty \pp(X_n = 1) < \infty$,
     thus by Borel Cantelli, almost surely $X_n = 1$ happens a finite number
     of times, thus $X_n \overset{a.s.}{\longrightarrow}0$.
     \newline
     For $\alpha \leq 1$, we have that $\sum_{n=1}^\infty \pp(X_n = 1) =
     \infty$, and $\sum_{n = 1}^\infty \pp(X_n = 0) = \infty$
     thus by Borel Cantelli, since $(\{ X_n = 1 \})_{n \geq 1}$ are
     independent and $(\{ X_n = 0 \})_{n \geq 1}$ are as well, we have 
     that almost surely $X_n = 1$ and $X_n = 0$ infinitely often, so almost
     surely it does not converge.
   \end{example}
  \begin{lemma}
      If $X_n \overset{\pp}{\longrightarrow} X$ and $X_n \overset{\pp}{\longrightarrow} Y$, then $X = Y$  almost surely.
  \end{lemma}
  \begin{proof}
    Fix $m \geq 1$, then $\pp(|X - Y| \geq 2/m) \leq \pp(|X_n - X| \geq 1/m) +  
    \pp(|X_n - Y| \geq 1/m)$, but as $n \to \infty$, we have that the terms in
    the right hand side converge to $0$, thus $\pp(|X - Y| \geq 2/m) = 0$,
    fomr which the result follows.
  \end{proof}
   \begin{lemma}
     [Subsequence Lemma]
     We have $X_n \overset{\pp}{\longrightarrow} X$ iff of every subsequence
     of $(X_n)$ we can extract a subsubsequence which converges $a.s$ to $X$.
     (a subsequence of $(X_n)$ is $(X_{\varphi(n)})$ with $\varphi$ an increasing
     function mapping the naturals to itself.)
   \end{lemma}
   \begin{proof}
      $\boxed{\Rightarrow}$ Let $\phi$ be a subsequence. Since $X_n
      \overset{\pp}{\longrightarrow} X$, we have $X_{\varphi(n)}
      \overset{\pp}{\longrightarrow} X$ so $\ee[\min(|X_{\varphi(k)} - X|, 1)]
      \underset{k \to \infty}{\longrightarrow} 0$

      \vspace{0.3em}

      \noindent
      Therefore we can find a subsequence $\psi$ such that $\forall n \geq 1$
      $\ee[min(|X_{\varphi(\psi(n))} - X|, 1)] \leq 1/2^n$. 
      Indeed, for $k$ sufficiently large we have $\ee[\min(|X_{\varphi(k)} - X|
      , 1)] \leq 1/2^n$. Then $\sum_{n=1}^\infty
      \ee[\min(|X_{\varphi(\psi(n))} - X|, 1)] < \infty$, which then implies
      $\ee[\sum_{n=1}^\infty
      \min(|X_{\varphi(\psi(n))} - X|, 1)] < \infty$, thus almost surely
      $\sum_{n=1}^\infty \min(|X_{\varphi(\psi(n))} -X|, 1) < \infty$, which
      is enough to conclude that $|X_{\varphi(\psi(n))} -X|$ 
      converges almost surely to $0$.

      \vspace{1em}
      \noindent
      $\boxed{\Leftarrow}$ Assume that $\forall \varphi$, $\exists \psi$ such
      that $X_{\varphi(\psi(n))} \overset{a.s.}{\longrightarrow} X$. Argue by
      contradiction, then $\ee[\min(|X_n - X|, 1)] \underset{n \to
      \infty}{\not\rightarrow}0$.

      \vspace{0.3em}
      \noindent

      Thus there exists $\varepsilon>0$ and a subsequence $\phi$ such that
      $\ee[\min(|X_{\phi(n)} - X|, 1)] \geq \varepsilon$. But by assumption,
      there exists a $\psi$ subsequence such that $X_{\varphi(\psi(n))}
      \overset{a.s.}{\longrightarrow} X$, thus $X_{\varphi(\psi(n))}
      \overset{\pp}{\longrightarrow} X$, thus $\ee[\min(|X_{\varphi(\psi(n))}
      - X|, 1)] \underset{n \to \infty}{\longrightarrow} 0$, which contradicts
      the first identity of this paragraph.
   \end{proof}

   \begin{application}
       Assume $X_n \overset{\pp}{\longrightarrow}X$ and $f$ continuous. Then
       $f(X_n) \overset{\pp}{\longrightarrow} f(X)$.
   \end{application}
   \begin{proof}
       Take any $\varphi$ a subsequence, then by the subsequence lemma there
       exists $\psi$ such that $X_{\varphi(\psi(n))}
       \overset{a.s}{\longrightarrow} X$, which implies
       $f(X_{\varphi(\psi(n))}) \overset{a.s.}{\longrightarrow} f(X)$, which
       in turn implies by the subsequence lemma the desired identity.
   \end{proof}
   \begin{example}
     [Flying Saucepans]
     Equip $[0, 1]$ with the Borel $\sigma-$field, and let $\lambda$ be the
     Lebesgue Measure. For $k\geq0$ and $0 \leq j \leq 2^k - 1$ define
     \[
       X_{2^k + j}(\omega) = \ind_{\left[\frac{j}{2^k}, \frac{j+1}{2^k} \right]}(\omega)
     .\] 
   Then $X_n \overset{\pp}{\longrightarrow} 0 $ as $\pp(|X_n| > \varepsilon)
     \leq 1/n$.

     \noindent
     But $\forall \omega \in [0, 1]$, there exists infinitely many $n \geq 1$
     such that $X_n(\omega) = 1$, so $X_n$ diverges almost surely.
   \end{example}
   \begin{example}
     [Spiky Cat]
     Take again $[0, 1]$
     Set $X_n(\omega) = 2^n \ind_{[0, 1/2^n]}(\omega)$ for $\omega \in[0, 1]$,
     then $X_n \overset{a.s}{\longrightarrow} 0$ but $\ee[X_n] = 1$, so $X_n$
     does not converge to $0$ by $L^1$.
   \end{example}
\end{document}
