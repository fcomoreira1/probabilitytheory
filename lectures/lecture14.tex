%! TEX root = ../main.tex
\documentclass[../main.tex]{subfiles}

\author{Francisco Moreira Machado}

\title{Lecture 14}

\begin{document}
In the example above, the portion of space where $X_n \neq 0$ becomes small, 
however its contribution to the expected value is constant. We have 
a probabilistic notion that prevents such spikes, wich is \underline{uniform
integrability}.

\vspace{0.5em}

\noindent
We saw that for $X \in L^1$ we have $\ee[|X| \ind_{|X| \geq x}] \underset{x \to
\infty}{\longrightarrow} 0$ by dominated convergence.
Uniform integrability extends this to a family of random variables.

\begin{definition}
  [Uniformly Integrable Family]
  A family $(X_i)_{i \in I}$ of integrable random variables is uniformly
  integrable if $\sup_{i \in I} \ee[|X_i| \ind_{|X_i| \geq x}] \underset{x \to
  \infty}{\longrightarrow}0$
\end{definition}

Equivalently, $\forall \varepsilon > 0, \exists x > 0$ such that $\forall i \in
I, \ee[|X_i| \ind_{|X_i| \geq x}] \leq \varepsilon$.

\begin{example}
    \hfill
    \begin{itemize}
      \item A finite family of $L^1$ random variables is UI by dominated
        convregence applied a finite number of times.
      \item If $Z \geq 0$ is integrable, then $\{ X \colon |X| \leq Z \} $ is
        UI. Indeed if $|X| \leq Z$, the $\ee[|X| \ind_{|X|\geq x}] \leq \ee[Z
        \ind_{Z \geq x}]$.
      \item If $(X_i)_{i \in I}$ is bonuded in $L^p$ for $p > i$ i.e., $\exists
        c>0$ such that $\forall i \in I$ $\ee[|X_i|^p] \leq C$, then $(X_i)$ is
        uniformly integrable. Indeed 
        \[
          \ee[|X_i|\ind_{|X_i| \geq x}] = 
          \ee\left[\frac{|X_i|}{|X_i|^p} |X_i|^p\ind_{|X_i| \geq x}\right] \leq
          \frac{\ee[|X_i|^p]}{x^{p-1}} \leq \frac{C}{x^{p-1}} 
        .\] 
    \end{itemize}
\end{example}

\begin{remark}
  By definition, a sequence $(X_n)_{n \geq 1}$ of $L^1$ random variables is UI
  if $$\sup_{n \geq 1} \ee[|X_n|\ind_{|X_n| \geq k}] \underset{k \to
  \infty}{\longrightarrow}0$$ But since it is a finite family of  $L^1$ random
  variables, this is equivalent to
  $$\limsup_{n \geq 1} \ee[|X_n|\ind_{|X_n| \geq k}] \underset{k \to
  \infty}{\longrightarrow}0$$
\end{remark}

\begin{theorem}
  [$\varepsilon-\delta$ condition]
  A family $(X_i)_{i \in I}$ of $L^1$ random variables is Uniformly Integrable
  iff 
  $(X_i)_{i \in I}$ is bounded in $L^1$ (i.e. $\sup_{i \in I} \ee[|X_1|] < A$)
  and $\forall \varepsilon > 0 \exists \delta > 0$ such that $\forall$ event $A$
  with $\pp(A) \leq \delta$, $\ee[|X_i| \ind_{A}] \leq \varepsilon$ for every $i
  \in I$
\end{theorem}
\begin{corollary}
  If $(X_i)_{i \in I}$ and $(Y_j)_{j \in J}$ are two families which are UI, then
  $\{ X_i + Y_j \colon i \in I, j \in J \} $ is UI.
\end{corollary}
\begin{proof}
  $\boxed{\Rightarrow}$
  Let $K > 0$ be such that $\ee[|X_i| \ind_{|X_i| \geq k}] \leq 1$ for every $i
  \in I$, then 
  $$
  \ee[|X_i|] = \ee[|X_i| \ind_{|X_i| \geq k}] + \ee[|X_i| \ind_{|X_i| \leq k}]
  \leq 1 + k
  $$ 
  so $(X_i)$ is bounded in $L^1$.

  \vspace{0.5em}

  \noindent Now we proceed for the $\varepsilon-\delta$ condition. Fix
  $\varepsilon > 0$. Let $K_{\varepsilon}$ be such that $\sup_{i \in I}
  \ee[|X_i| \ind_{|X_i| \geq K_{\varepsilon}}] \leq \varepsilon$, then taking
  $\delta = \varepsilon / K_{\varepsilon}$ we get for $\pp(A) \leq \delta$
  \[
    \ee[|X_i| \ind_A] = 
    \ee[|X_i| \ind_A \ind_{|X_i| \geq K_{\varepsilon}}] +
    \ee[|X_i| \ind_A \ind_{|X_i| \geq K_{\varepsilon}}] 
    \leq \varepsilon + K_{\varepsilon}\pp(A) \leq 2 \varepsilon
  .\] 

  \vspace{1em}
  $\boxed{\Leftarrow}$ Fix $\varepsilon > 0, \delta > 0$ such that the condition
  holds. Let $k > 0$ be such that $\sup_{i \in I}\ee[|X_i|] \leq K \delta$. Then
  by Markov's inequality
  \[
    \pp(|X_i| \geq k) \leq \frac{\ee[|X_i|]}{K} \leq \delta 
  \] 
  Thus we can just apply the $\varepsilon-\delta$ condition with $A = \{ |X_i|
  \geq k \} $ to get the desired result.
\end{proof}

UI bridges the gap between convergence in $\pp$ and convergence in $L^1$

\begin{theorem}
  [Super Dominated Convergence]
  Let $(X_n)$ be integrable real-valued random variables, $X$ a real valued
  random variable then the following conditions are equivalent
  \begin{enumerate}
    \item $X \in L^1$ and $X_n \overset{L^1}{\longrightarrow} X$
    \item $X_n \overset{\pp}{\longrightarrow} X$ and $(X_n)_{n \geq 1}$ is UI.
  \end{enumerate}
\end{theorem}
(The name comes from the fact that $\{ X \colon |X| \leq Z \} $ with $Z \geq 0$
integrable is a UI family: it implies dominated convergence).

\begin{proof}
  $\boxed{1. \Rightarrow 2.}$ We know that  $X_n \overset{L^1}{\longrightarrow}
  X$ implies $X_n \overset{\pp}{\longrightarrow}X$. To show that $(X_n)_{n \geq
  1}$ is UI by the corollary, it suffices to show that $(X_n - X)_{n \geq 1}$ is
  UI.
  \vspace{0.4em}

  \noindent
  To do this, fix $\varepsilon>0$ and choose $n_0$ such that $n \geq n_0$
  implies $\ee[|X_n - X|] \leq \varepsilon$. Let $k_0$ be such that $k \geq k_0$
  implies $\max _{1 \leq i \leq n_0} \ee[|X_i - X| \ind_{|X_i - X| \geq k}] \leq
  \varepsilon$.

  \vspace{0.4em}
  \noindent
  Thus $\forall n \geq 1$, $\ee[|X_n - X| \ind_{|X_n - X| \geq k}] \leq
  \varepsilon$ for $k \geq k_0$.

  \vspace{1em}
  \noindent
  $\boxed{2. \Rightarrow 1.}$ We first show that $X \in L^1$. Since $X_n
  \overset{\pp}{\longrightarrow} X$, there exists a subsequence $\psi$ such
  that $X_{\psi(n)} \overset{a.s.}{\longrightarrow} X$.

  \vspace{0.4em}
  \noindent
  Thus by Fatou's Lemma 
  \[
    \ee[|X|] = \ee[\liminf_{n \to \infty} |X_{\psi(n)}|] \leq \liminf_{n \to
    \infty} \ee[|X_{\psi(n)}] < \infty
  \] 
  Now we show that $X_n \overset{L^1}{\longrightarrow} X$. Since $X \in L^1$, we
  have $(X_n - X)$ is UI by the corollary. Now fix $\varepsilon > 0$ and let
  $\delta > 0$ be such that the $\varepsilon-\delta$ condition holds.

  \vspace{0.4em}
  \noindent
  Then for $n$ sufficiently large $\pp(|X_n - X| \geq \varepsilon) \leq \delta$
  because $X_n \overset{\pp}{\longrightarrow} X$. Thus

  \[
    \ee[|X_n - X|] = \ee[|X_n - X|\ind_{|X_n - X| < \varepsilon }] + 
    \ee[|X_n - X|\ind_{|X_n - X| \geq \varepsilon }]
    \leq \varepsilon + \varepsilon
  \] 
  finishing the proof.
\end{proof}
\begin{remark}
    Existence of a sequence of iid random variables. We have implicitly used the
    following theorem so far
\begin{theorem}
  Let $\mu$ be a probability distribution on $\rr$. There exists a sequence
  $(X_n)_{n \geq 1}$ of iid random variables with law $\mu$.
\end{theorem}
    This is related to the existence of product measures on infinite product
    spaces (see lecture notes).
\end{remark}

\section{Conditional Expectation}
\subsection{Discrete Setting}
\underline{Goal:} see how the knowledge of information modifies probability
measures. Here we will "just" define the conditional expectation of random
variables given a $\sigma-$field.

\vspace{0.5em}
\noindent
Let $(\Omega, \mathcal{A}, \pp)$ be a probability space. Take $B \in
\mathcal{A}$ with $\pp(B) > 0$. We can define
\[
  \pp(A | B) = \frac{\pp(A \cap B)}{\pp(B)} 
\]
for $A \in \mathcal{A}$. 

\vspace{0.5em}
\noindent
$\pp(\cdot \vert B)$ defines a probability measure called the conditional
probability given the EVENT B.

\vspace{0.5em}
\noindent
Similarly for $X \in L^1$ we define 
\[
  \ee[X | B] = \frac{\ee[X \ind_{B}]}{\pp(B)} 
.\] 
\underline{Interpretation:} Average value of $X$ when $B$ occours. $\ee[X | B]$
is the expectation of $X$ in $(\Omega, \mathcal{A}, \pp(\cdot | B))$.

\vspace{1em}
\noindent
Now let $Y \colon \Omega \to E$ be a random variable with E countable. We want
to define $\ee[X | Y]$. From before, we have $\ee[X | Y = y] = \cfrac{\ee[X \ind_{Y =
y}]}{\pp(Y = y)}$ for every $y$ with $\pp(Y = y) > 0$. 

\vspace{0.5em}
\noindent
Thus we naturally set $\phi(y) \colon E \to \rr$ to $y \mapsto \ee[X | Y = y]$
if $\pp(Y = y) > 0$ and $0$ otherwise. Moreover, $\phi(Y)$ is itself a random
variable which is $\sigma(Y)-$measurable.

\vspace{0.3em}\noindent
In other words: $\ee[X | Y] (\omega) = \phi(Y(\omega))$
\end{document}
