%! TEX root = ../main.tex
\documentclass[../main.tex]{subfiles}

\title{Lecture 25}

\begin{document}
\begin{corollary}
  Let $X_n, X$ be $\zz-$valued r.v. then $X_n \overset{(d)}{\longrightarrow}
  X$ iff $\forall k \in \zz$, $\pp(X_n = k) \underset{n \to
  \infty}{\longrightarrow} \pp(X = k)$.
\end{corollary}
\begin{proof}
  $\boxed{\implies}$ Fix $k \in \zz$ and take $f_k(x) = \max(1 - |x - k|, 0)$
  which is continuous and bounded, thus $\ee[f_k(X_n)] \to \ee[f_k(X)]$ thus
  $\pp(X_n = k) \to \pp(X = k)$

  \vspace{0.5em}
  \noindent
  $\boxed{\impliedby}$ Take $f \in \mathcal{C}_c(\rr^d)$ and assume $\forall k
  \in \zz$, $\pp(X_n = k) \to \pp(X = k)$.

  Write $\ee[f(X_n)] = \sum_{j \in \zz} \pp(X_n = j)f(j)$ and $\ee[f(X)] = \sum_{j
  \in \zz} \pp(X = j)f(j)$. To prove convergence, notice that as $f$ has
  compact support, the two sums can be indexed by a finite set ($\zz \cap
  support(f)$). Then we can interchange the limit and sum over a finite set
  and conclude.
\end{proof}
\begin{application}
  Take $\lambda > 0$, $X_n \sim Bin(n , \frac{\lambda}{n} )$, then $X_n
  \overset{(d)}{\longrightarrow} Poi(\lambda)$.
  (This is the reason why the Poisson distribution is used to model rare
  events)
\end{application}

\subsection{Characteristic functions and Lévy's theorem}

Characteristic functions are defined as expectations of $\mathbb{C}-$valued
rnadom variables. When $Z$ is a $\cc-$valued r.v., $\ee[|Z|] < \infty$, we say
$Z$ is integrable and define $\ee[Z] = \ee[Re Z] + i \ee[Im Z]$.

\begin{definition}
    The characteristic functionn of a $\rr^d-$valued r.v. X is defined by 
    \begin{align*}
      \varphi_X \colon &\rr^d \to \cc \\
      &u \mapsto \ee[e^{i \langle X | u \rangle}]
    .\end{align*} 
\end{definition}
\begin{remark}
    $\varphi_X$ is well defined as $e^{i \langle X | u \rangle}$ is an
    integrable r.v. because its absolute value is $1$.
\end{remark}

\begin{example}
  If $X \sim Poi(\lambda)$, for $u \in \rr$, $\varphi_X(u) = \ee[e^{iXu}] =
  e^{\lambda(e^{iu} - 1)}$
\end{example}

\begin{remark}
  By the transfer theorem, $\varphi_X(u) = \int_{\rr^d} e^{i \langle x | u
  \rangle} \pp_X(dx)$ for $u \in \rr^d$. In measure theoretical terms,
  $\varphi_X$ is the Fourier transform of $\pp_X$.
\end{remark}

\begin{proposition}
    $\varphi_X$ always satisfies the following:
    \begin{itemize}
      \item $\varphi_X(0) = 1$.
      \item $\varphi_X(-u) = \overline{\varphi_X(u)}$ for $u \in \rr^d$.
      \item $|\varphi_X(u)| \leq \ee[|e^{i \langle X | u \rangle}|] = 1$ for
        $u \in \rr^d$
      \item $|\varphi_X(u + h) - \varphi_X(u)| \leq \ee[ |e^{i \langle X | h} -
        1|]$ for $u, h \in \rr^d$. In particular, $\varphi_X$ is uniformly
        continuous.
    \end{itemize}
\end{proposition}

In what follows Guassian r.v.s play a crucial role.
\begin{example}
  Take $X \sim N (m, \sigma^2)$, $\varphi_X(u) = e^{imu - \frac{\sigma^2u^2}{2}
  }$ for $u \in \rr$.
\end{example}
\begin{sketch}
  Using properties of the Gaussian, it is enough to show the result for $N(0,
  1)$. Since $\varphi_X(u) = \varphi_X(-u) = \overline{\varphi_X(u)}$ as $N(0,1) = -N(0,1)$
  we have that $\varphi_X(u) \in \rr$. 

  Thus $$\varphi_X(u) = \int_{\rr} \frac{1}{\sqrt{2\pi}} \cos(xu)
  e^{-x^2/2} dx$$ To compute this, the idea is to see that $\varphi_X$ solves a
  differential equation.
  
  Indeed, we have 
  \[
    \varphi_X'(u) = \frac{1}{\sqrt{2\pi}} \int_{\rr} -x \sin(xu)e^{-x^2 /
    2}dx.
  \] (Justification: use the theorem that allows to differentiate an integral
  depending on a parameter, which is possible because $|-x\sin(xu) e^{-x^2}|
  \leq x e^{-x^2/2}$ which is integrable)

  Now, integration by parts gives
  \[
    \varphi_X'{u} = -\int_{\rr} \frac{1}{\sqrt{2\pi}} e^{-x^2/2} u \cos(ux)dx
  \]
  hence $\varphi_X'(u) = -u \phi_X(u)$. But this system has the initial
  condition $\varphi_X(0) = 1$, which one can solve to obtain $\varphi_X(u) =
  e^{-u^2/2}$.
\end{sketch}

\begin{remark}
  If $g_{\sigma}(x) = \frac{1}{\sigma\sqrt{2\pi}} e^{\frac{-x^2}{2 \sigma^2} }$ this
  example shows that $g_{\sigma}(z) = \frac{1}{\sigma \sqrt{2\pi}} \int_{\rr}
  e^{iuz} g_{1/\sigma}(u) du$ for every $z \in \rr$.
\end{remark}

\begin{theorem}
    Let $X, Y$ be r.v. in $\rr^d$, $\varphi_X = \varphi_Y$ iff $X$ and $Y$ have the
    same law.
\end{theorem}
\begin{proof}
    To simplify, assume $d = 1$ and that $X, Y$ are defined in the same
    probability space.

    $\boxed{\impliedby}$ If $\pp_X = \pp_y$, this is directly true by the
    transfer theorem.

    $\boxed{\implies}$
    \underline{Idea:} use a small gaussian perturbation. More precisely let
    $Z_n$ be a $N(0, 1/n)$ r.v. $\indep X, Y$.

    The idea now is to show $\varphi_X = \varphi_Y$ implies $X + Z_n
    \overset{(d)}{=} Y + Z_n$ $(\star)$. Indeed, assume this hows, let us see
    how we can conclude.

    $\ee[Z_n^2] = 1/n$, so $Z_n \overset{L^2}{\longrightarrow} 0$ thus $Z_n
    \overset{\pp}{\longrightarrow} 0$. Thus $X + Z_n
    \overset{\pp}{\longrightarrow} X + 0$. Thus $X + Z_n
    \overset{(d)}{\longrightarrow} X$, similarly $Y + Z_n
    \overset{(d)}{\longrightarrow} Y$, from which the conclusion holds.

    Now we must go back to show $(\star)$. For this, we show that for $F
    \colon \rr \to \rr_+$ measurable, $\ee[F(X + Z_n)] = \ee[F(Y + Z_n)]$.

    We will prove that $\ee[F(X + Z_n)] = \ee[F(Y + Z_n)]$ only depends on
    $\phi_X$. Let's compute
    \begin{align*}
      \ee[F(X + Z_n)] &= \int_{\rr} \pp_X (dx) \left( \int_{\rr} F(x + z)
      \pp_{Z_n}(dz) \right)  \tag*{Transfer+Fubini}\\
      &= \ee[\int_{\rr} F(x + z) g_{1/\sqrt{n}}(z) dz] \\
      &= \ee[\int_{\rr} F(z) g_{1/\sqrt{n}}(z - X)dz] \\
      &= \int_{\rr} F(z)dz \ee[ g_{1/\sqrt{n}} (z - X)] \tag*{Fubini-Tonelli} 
    .\end{align*} 
    Now we look at $\ee[g_{1/\sqrt{n}}(z - X)]$. But we know that
    $g_{1/\sqrt{n}}(z - X) =
    \frac{\sqrt{n}}{\sqrt{2\pi}} \int_{\rr} e^{iu(z - X)} g_{\sqrt{n}}(u) du$.

    Taking the expectation and using Fubini-Lebesgue
    \[
      \ee[g_{\sqrt{\frac{1}{n}}} (z - X)] = \frac{\sqrt{n}}{\sqrt{2\pi}}
      \int_{\rr} \ee[e^{-iux}] e^{-iuz} g_{\sqrt{n}}(u)du
    \] 
    but $\ee[e^{-iux}] = \varphi_X{-u}$, thus $\ee[F(X + Z_n)]$ only depends
    on $\varphi_X = \varphi_Y$ and the equality holds.  
  
\end{proof}

\end{document}
