%! TEX root = ../main.tex
\documentclass[../main.tex]{subfiles}

\usepackage[font, sexy]{moreira}
\usepackage{marginnote}
\reversemarginpar

\author{Francisco Moreira Machado}

\title{Lecture 4}

\begin{document}

If $X$ and $Y$ are two r.v., how ca we check if they have the same law, i.e. if $\pp_X =
\pp_Y$?
How can one characterize a probability measure.

\vspace{0.8em}
\noindent
\textbf{\sffamily Nice Case} $E$ is countable. Indeed if $X \colon (\Omega, \mathcal{A})
\to \mathcal{P}(E)$ is a r.v. with $E$ countable, its law is characterized by the values
$$\pp_{X}(x) = \pp_X(\{ x \}) = \pp(X = x) \text { with } x \in E$$
with this, for $A \subset E$, $\pp_X(A) = \sum_{x \in A} \pp(\{ x \})$.
In particular, $\pp(X = z) = \pp(Y = z) \;\forall z \in E$ implies $\pp_X = \pp_Y$.

\vspace{1em}

When $E = \rr$, cumulative distribution functions (cdf) are useful.

\begin{definition}
  [cdf]
  If $X \colon (\Omega, \mathcal{A}) \to (\rr, \mathcal{B}(\rr))$ is a r.v., its cdf is the
  function $F_X \colon \rr [0, 1]$ defined by 
  \[
    F_X(x) = \pp(X \leq x) = \pp(\{ \omega \in \Omega \colon X(\Omega) \leq x\}) = \pp_X(]
    -\infty, x])
  .\]
\end{definition}
\newpage

\begin{example}
  [Bernoulli Distribution]
    Bernoulli random variable $\pp(X = 0) = 1/4$, $\pp(X = 1) = 3/4$.
    \[
    F_X(x) = \begin{cases}
      0 & x < 0 \\
      1/4 & 0 \leq x < 1 \\
      1 & 1 \leq x
    \end{cases}
    \] 
\end{example}

\begin{example}
  [Uniform Distribution]
  Assume that the law of $X$ is the Lebesgue measure on $[0, 1]$
\end{example}

\begin{proposition}
  The following characterize a random variable.
  \begin{enumerate}
      \item Let $X$ be a $\rr-$valued r.v. Then $F_X$ is non-decreasing, $\lim_{-\infty} F_X = 0$,
    $\lim_{\infty} F_X = 1$, $F_X$ is right-continuous
    
      \item If $F_Y = F_X$ then $\pp_X = \pp_Y$

      \item {\sffamily (Lebesgue-Stieltjes)} If $F \colon \rr \to [0, 1]$ satisfies the
        properties of 1., then there exiss a $\rr-$valued r.v. $X$ s.t. $F_X = F$
  \end{enumerate}
\end{proposition}
\begin{proof}
  First, it is clear that a cdf must be non-decreasing. Due to that, we know that $F_x$ is
  monotone and bounded, and thus it has its limits well defined. 
  \vspace{0.5em}

  We can define $A_n = \bigcap_{k = 1}^n ]-\infty, -k]$, which is a decreasing sequence, thus 
  $\pp_X(\lim_{n \to \infty} A_n) = \lim_{n \to \infty} \pp_X(A_n) = \lim_{n \to \infty}
  F_x(-n)$, from which we can conclude. The other limit is analogous.

  \vspace{0.5em}

  Now for right continuity, we define very similar sets $A_n = \bigcap_{k = 1}^n ]-\infty, x +
  1/k]$ and proceed similarly.

  \vspace{0.5em}

  To prove 2., notice that $\{ ]-\infty, x] \colon x \in \rr \} $ is a generating $\pi-$system
  of $\mathcal{B}(\rr)$, thus by the corollary of the Dynkin lemma, if $\pp_X, \pp_Y$ coincide
  in this set, they are equal.

  \vspace{0.6em}

  Take $\Omega =  ]0, 1[$ equiped with $\mathcal{A} = \mathcal{B}(]0, 1[)$. For $\omega \in ]0,
  1[,$ and $\pp = \lambda$ set $X(\omega) = \inf \{ t \in \rr \colon F(t) \geq \omega\} $ (called the
  right-continuous inverse of $F$).

  \noindent Then $X$ is measurable and $X(\omega) \leq x \iff x \leq F(X)$
  \marginnote{Ex. $\rightarrow$}

  Then $F_X(x) = \pp(X \leq x) = \pp(\omega \leq F(X)) = \pp(\{ \omega \in \Omega \colon \omega
    \leq F(x)\}) = F(x) $

\end{proof}

\begin{remark}
  \marginnote{Ex. $\rightarrow$} Similarly, one can show that 
  $$F_X(x) - F_X(x-) = \pp(X = x)$$
  In particular, if $F_X$ is continuous, $\pp(X = x) = 0 \;\forall x \in \rr$.
\end{remark}

\begin{notation}
    If $f \colon E \to (F, \mathcal{F})$ is a function we set $\sigma(f) = \{ f^{-1}(B) \colon
    B \in \mathcal{F}\} $. It is a $\sigma-$field (exercise) \marginnote{Ex. $\rightarrow$}
    called the $\sigma-$field generated by $f$.

    \vspace{0.5em}

    \noindent Similarly if $(f_i)_{i \in I}$ is a collection of functions $f_{i}\colon E \to
    (F_i, \mathcal{F}_i)$ we define $\sigma(f_{i}, i \in I) = \sigma(\{ f_{i}^{-1}(B_i) \colon
    B_i \in \mathcal{F}_i, i \in I\}) $ to be the $\sigma-$field generated by $(f_i)_{i \in
  I}$.
\end{notation}

\noindent \underline{\sffamily Interpretation in Probability:} $\sigma(X)$ represents the 
"information" / "observable sets" one has access to by looking at the the values of $X$.

\begin{example}
    $f \colon \rr \to \rr$ defined by $f(x) = x^2$. Then $\sigma(f) = \{ A \in \mathcal{B}(\rr)
    \colon A = -A\} $.
\end{example}
\begin{proposition}
    \hfill
    \begin{enumerate}
      \item Let $f \colon E \to (F, \mathcal{F})$ be a function. Then $\sigma(f)$ is the
        smallest $\sigma-$field on $E$ such that $f$ is measurable.
      \item Let $(f_i)_{i \in I}$ with $f_i \colon E \to (F_i, \mathcal{F}_i)$ be a collection
        of functions, then its sigma field is the smallest $\sigma-$field on $E$ such that all
        the $f_i$ are measurable.
    \end{enumerate}
\end{proposition}
\begin{proof}
    We check that $f\colon(E, \sigma(f)) \to (F, \mathcal{F})$ is measurable. This is indeed
    true by definition of $\sigma(f)$.
    \vspace{0.3em}
    Assume now that $f\colon(E, \mathcal{E}) \to (F, \mathcal{F})$ is measurable. We now show
    $\sigma(f) \subset \mathcal{E}$. Indeed, since $f$ is measurable, $\forall B \in
    \mathcal{F}, f^{-1}(B) \in \mathcal{E}$, thus $\sigma(f) \subset \mathcal{E}$.
    
    The second part is left as exercise. \marginnote{Ex. $\rightarrow$}
\end{proof}

\begin{proposition}
    Let $E, F$ be metric spaces. Let $f \colon E \to F$ be continuous, then \\ $f \colon (E,
    \mathcal{B}(E)) \to (F, \mathcal{B}(F))$ is measurable.
\end{proposition}
\begin{proof}
  $\forall O \subset F$ open, we have that $f^{-1}(O)$ is open by continuity of $f$, thus
  $f^{-1}(O) \in\mathcal{B}(E)$. Thus for $\mathcal{C} = \{ O \colon O \subset F, \text{ open}
  \} $, which is a generating system of $\mathcal{B}(F)$, we have that $\forall O \in
  \mathcal{C}, f^{-1}(O) \in \mathcal{B}(E)$. Thus $\forall B \in \sigma(\mathcal{C}) =
  \mathcal{B}(F), f^{-1}(B) \in \mathcal{B}(E)$.
\end{proof}

\subsection{Product $\sigma-$fields and families of functions}

Product $\sigma-$fields are needed when considering pairs of random variables, and more
generally families of r.v.

\vspace{0.5em}
\noindent \underline{\sffamily Idea:} View a collection $(X_i)_{i \in I}$ of random variables
as ONE random variable.

\begin{definition}
  [Product $\sigma-$field]
  Let $(E_i, \mathcal{E}_i)_{i \in I}$ be a measurable space. Set $E = \prod_{i \in I} E_i$.
  An element $x \in E$ is written as $(x_i)_{i \in I}$ for $i \in I$ set $\Pi_i \colon E \to
  E_i$ is the projection onto the $i-$th coordinate called the canonical projections.
\end{definition}

\begin{example}
    $E = \{ 0, 1 \} ^{\nn}$, then $\Pi_{j} \colon E \to \{ 0,1 \} , \Pi_j((x_i)_{i \in I}) =
    x_j$.
\end{example}
\begin{example}
  $E = \prod_{i \in [0, 1]} \rr = \{ f \colon [0, 1] \to \rr \} $ is the space of functions
  from $[0, 1]$ to $\rr$.
\end{example}

\begin{definition}
  [Product $\sigma-$field or Cylinder $\sigma-$field]
  We define $\otimes \mathcal{E}_i = \sigma(\Pi_i \colon i \in I)$ to be the smallest
  $\sigma-$field on $\prod_{i \in I}E_i$ for which the canonical projections are measurable.
\end{definition}
\begin{definition}
  [Cylinder Sets]
  Sets of the form $\Pi^{-1}_{i_1}(A_1) \cap \ldots \Pi_{i_k}^{-1}(A_k)$ with $i_1, \ldots, i_k
  \in I$, $A_1 \in \mathcal{E}_{i_1},\ldots, A_k \in \mathcal{E}_{i_k}$ are called cylinders.
  They are a generating $\pi-$system of $\otimes_{i \in I}\mathcal{E}_i$
\end{definition}

\begin{proposition}
    If $|I| = n$ then $\otimes _{i=1}^n \mathcal{E} = \sigma(\{ A_1 \times \ldots \times A_n \colon A_i \in
    \mathcal(E)_i \} $
\end{proposition}
\begin{proof}
  Set $\mathcal{E} = \sigma(A_1\times \ldots \times A_n \colon A_i \in \mathcal{E}_i)$. We show
that $\mathcal{E}$ is the smallest $\sigma-$field on $E_1 \times \ldots \times E_n$ for which
the $\Pi_i$'s are measurable.

$\Pi_i \colon (E, \mathcal{E}) \to E_i$ is measurable because for $B \in \mathcal{E}_i$
$\pi_{i}^{-1}(B) = E_1 \times \ldots E_{i -1} \times B \times E_{i+1} \times E_n \in
\mathcal{E}$. So $\Pi_i$ is measurable $\forall i$, then for $A_i \in \mathcal{E_i}$ $A_1
\times \ldots A_n = \Pi_{1}^{-1}(A_1) \cap \ldots \Pi_n^{-1}(A_n) \in \mathcal{E}$ by
measurability. Hence $\sigma(\{ A_1 \times \ldots \times A_n \colon A_i \in \mathcal{E} \} $ is
in the $\sigma-$field.
\end{proof}

\end{document}
