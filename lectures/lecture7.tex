%! TEX root = ../main.tex
\documentclass[../main.tex]{subfiles}

\usepackage[font, sexy]{moreira}
\usepackage{marginnote}
\reversemarginpar

\author{Francisco Moreira Machado}

\title{Lecture 7}

\begin{document}
 \begin{application}
   Let $X, Y, Z, T$ be $\indep$ random variables, then $X + Z$ and $YT$ are $\indep$
 \end{application} 
 \begin{proof}
   Indeed, $X, Z, Y, T$ are $\indep$ ($\indep$ is preserved under permutation).
   then we apply the Coalition Principle to get that $(X, Z)$ and $(Y, T)$ are 
   independent. Moreover, by the Composition Principle, we have that $f_1(X, Z)$ and 
   $f_2(Y, T)$ are independent if we pick two measurable functions $f_1(x, z) = x + z$ and 
   $f_2(y, t) = yt$.
 \end{proof}

 \begin{lemma}
   The two random variables $(X_i)_{i \in I}$ and $(Y_i)_{i \in I}$ with values in 
   $\Pi_{i \in I} E_i$ and $\Pi_{i \in I} F_i$ are $\indep$ iff
   
   $$\forall i_1, \ldots, i_k \in I, \forall j_1, \ldots, j_l \in
   J, (X_{i_1}, X_{i_2}, \ldots, X_{i_k}) \indep (Y_{j_1}, \ldots,
   Y_{j_l})$$
 \end{lemma}

 \begin{definition}
   If $(X_i)_{i \geq 1}$ are random variables we set $B_n = \sigma(X_K
   \colon k \geq n)$ and $B_{\infty} = \bigcap_{n \geq 1}B_n$, which
   is a $\sigma-$field called the \underline{tail $\sigma-$field}.
 \end{definition}

 Intuitively $B_{\infty}$ represents information that does not
 depend on a finite number of random variables.

 \begin{example}
   If $(X_i)_{i \geq 1}$ are $\rr-$valued rv. Set $S_n = X_1 + \ldots
   + X_n$ then $\{ \sup_{n \geq 1} S_n = +\infty \} \in B_{\infty} $
 \end{example}

 \begin{theorem}
   [Kolmogorov $0-1$ law]
   Assume that $(X_i)_{i \geq 1}$ are $\indep$ then $\forall A \in
   B_{\infty}, \pp(A) = 0$ or $1$.
 \end{theorem}
 \begin{proof}
   Set $\mathcal{D}_n = \sigma(X_1, \ldots., X_n)$, then $\mathcal{D}_n
   \indep B_{n+1}$. Hence $\mathcal{D}_n \indep B_{\infty}$ because
   $B_{\infty} \subset B_{n+1}$.
   Thus $\forall A \in \bigcup_{n = 1}^\infty \mathcal{D}_n$, $\forall
   B \in B_{\infty}$, $\pp(A \cap B) = \pp(A)\pp(B)$. But $\bigcup_{n
   \geq 1} \mathcal{D}_n = \bigcup_{n \geq 1} \sigma(X_1,\ldots,
   X_n)$ is a generating $\pi-$system of $\sigma(X_i \colon i\geq 1)$.
   Thus
   \[
     \forall A \in \sigma(X_i\colon i \geq 1), \forall B \in
     B_{\infty}, \pp(A \cap B) = \pp(A) \pp(B)
   .\] 

 Finally, observe that $B_{\infty} \subset \sigma(X_n \colon n \geq
   1)$, thus $\forall A, B \in B_{\infty}$, $\pp(A \cap B) =
   \pp(A)\pp(B)$, from which we conclude that $\pp(A) = \pp(A)^2$
   $\forall A \in B_{\infty}$, finishing the proof.
 \end{proof} 

 \subsection{Real-valued random-variables}

  \begin{proposition}
      Let $f_n \colon (E, \mathcal{E}) \to (\overline{\rr},
      \mathcal{B}(\overline{\rr})$ be measurable functions where
      $\overline{\rr} = \rr \cup \{ \pm \infty \} $ with $d(x,y) =
      |\arctan x - \arctan y|$. Then $\sup_{n} f_n$ i.e. the function
      $x \mapsto \sup_n f_n(x)$, $\inf_n f_n$, $\limsup_{n} f_n$,
      $\liminf_n f_n$ are all measurable from $(E, \mathcal{E})$ to
      $(\overline{\rr}, \mathcal{B}(\overline{\rr})) $
  \end{proposition}

  \begin{proof}
      Let us show for $f = \sup f_n$. 
      \vspace{0.5em}

      $\sup_{n \geq 1} x_n \leq a \iff \forall n \geq 1, x_n \leq a$.
      Thus $\forall a \in \rr$, $f^{-1}([-\infty, a]) = \bigcap_{n
      \geq 1} f_n^{-1}([-\infty, a]) \in \mathcal{E}$ because $f_n$ is
      measurable.
      \vspace{0.5em}

      Since $([-\infty, a] \colon a \in \rr)$ generates
      $\mathcal{B}(\overline{R}),$ this shows that $f$ is measurable.
  \end{proof}
  \begin{definition}
    [Simple Function]
    A simple function $f\colon (E, \mathcal{E}) \to (\rr,
    \mathcal{B}(\rr))$ is a measurable function which takes a finite
    number of values. Equivalently $f$ can be written 
    \[
    f = \sum_{i = 1}^n a_i \ind_{A_i}
    \]
    with $a_i \in \rr$ and $A_i \in \mathcal{E}$. It can be uniquely
    written if we suppose $A_i$ are pairwise disjoint and we order the
    $a_i$.
  \end{definition}

  \begin{theorem}
    Let $f\colon (E, \mathcal{E}) \to (\rr^+, \mathcal{B}(\rr^+))$ be
    measurable. There exists a sequence $(f_n)$ of simple measurable
    functions $(E, \mathcal{E}) \to (\rr^+, \mathcal{B}(\rr^+))$
    such that
    $\forall x \in E$ the sequence $(f_n(x))_{n \geq 1}$ is weakly
    increasing and converges to $f(x)$.
  \end{theorem}

  This is a powerful tool to show properties for general functions.
  First we check the property for simple functions then conclude by
  approximations.

  \begin{proof}
      \hfill

      \underline{Step 1} Approximate the identity function. To do so,
      just take $\phi_n(x) = \min \left( \frac{1}{2^n} \left\lfloor
      2^n x \right\rfloor, n  \right) $, which only takes finitely
      many values.
      \vspace{0.5em}

      \underline{Step 2} Just take $f_n = \phi_n \circ f$.
  \end{proof}

  \begin{application}
    [Doob-Dynkin Lemma]
    Let $f\colon (E, \mathcal{E}) \to (F, \mathcal{F})$ and $g \colon
    (E, \sigma(f)) \to (\rr, \mathcal{B}(\rr))$ be measurable. Then
    $\exists h \colon (F, \mathcal{F}) \to (\rr, \mathcal{B}(\rr))$
    such that $g = h \circ f $
  \end{application}
  \underline{In Probabaility: } A $\sigma(X)-$measurable rv is just a
  function of $X$.
  \begin{remark}
    If $g = h \circ f$ then $g$ is $\sigma(f)-$ measurable since 
    \[
      g^{-1}(B) = (h \circ f)^{-1}(B) = f^{-1}(h^{-1}(B)) \in
      \sigma(f)
    .\] 
  \end{remark}
  \begin{proof}
    Assume $g \geq 0$ by decomposing $g = max(g, 0) + max(-g, 0)$.

    Now, consider the case $g = \ind_A$ with $A \in \sigma(f)$, then
    $A = f^{-1}(B)$ with $B \in \mathcal{F}$. we then take $h =
    \ind_B$, from which it follows.

    By linearity, the statement holds for any simple function, so now
    we can conclude by using the fact that we can write $g$ as a limit
    of simple functions $g_n = h_n \circ f$ and build $h$ to be the
    limit of $h_n$, then the desired result holds.
  \end{proof}
\end{document}
