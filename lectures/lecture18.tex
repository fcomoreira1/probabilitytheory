%! TEX root = ../main.tex
\documentclass[../main.tex]{subfiles}

\title{Lecture 18}

\begin{document}
\begin{proposition}
  [Discrete Stochastic Calculus (You can't trick the game)]
  A sequence $(H_n)_{n \geq 1}$ of real-valued random variables is called
  predictable if $\forall n \geq 1$, $H_n$ is bounded and $\mathcal{F}_{n-1}$
  measurable.

  For a sequence $(M_n)_{n \geq 0}$ we define $(H \cdot M)_{m} = \sum_{k=1}^n
  H_k(M_k - M_{k-1})$.
  \begin{itemize}
    \item If $(M_n)_{n \geq 0}$ is a martingale, then $(H \cdot M)_n$ is a
      martingale. In particular $\ee[(H \cdot M)_n] = 0$
    \item If $(M_n)$ is a sub/supermartingale and $H_n \geq 0 \forall n \geq 1$ 
      then  $(H \cdot M)_n)$ is a sub/supermartingale.
  \end{itemize}
\end{proposition}
\underline{Interpretation:} If $M_n$ represents the wealth of a player at time
$n$, $M_{n+1} - M_n$ represents the amount "won" at time $n$, and
$H_{n+1}(M_{n+1} - M_n)$ represents the amount won if the player had multiplied
by $H_{n+1}$ the bet at time $n$.

\begin{proof}
    \hfill
    \begin{itemize}
      \item $(H \cdot M)_{n} \in L^1(\Omega, \mathcal{F}_n, \pp)$ by definition.
        We check that $\ee[(H \cdot M)_{n+1} - (H \cdot M)_{n} | \mathcal{F}_n]
        = 0$ $\forall n \geq 0$.

        Indeed, $\ee[(H \cdot M)_n | \mathcal{F}_n] = (H \cdot M)_{n}$ so this
        implies $\ee[(H \cdot M)_{n+1} | \mathcal{F}_n] = (H \cdot M)_{n}$.

        Thus it suffices to check the identity above
        \[
          \ee[(H \cdot M)_{n+1}- (H \cdot M)_n | \mathcal{F}_n] =
          \ee[H_{n+1}(M_{n+1} - M_n) | \mathcal{F}_n] = H_{n+1} \ee[M_{n+1} -
          M_n | \mathcal{F}_n] = 0
        .\] 
    \item It goes similarly. \marginnote{Ex. $\rightarrow$}
    \end{itemize}
\end{proof}   

\subsection{The (sub/super)martingale a.s. convergence theorem}
Recall that a family $(X_i)_{i \in I}$ of random variables is bounded in $L^1$
if $\sup_{i \in I} \ee[|X_i|] < \infty$.

\begin{theorem}
  [Doob]
  Let $(M_n)$ be a (sub/super)martingale \textbf{bounded in $L^1$}. Then $(M_n)$
  converges a.s. to some real valued random variable $M_{\infty}$ with
  $\ee[|M_{\infty}|] < \infty$.
\end{theorem}
\begin{corollary}
    A non-negative supermartingale or martingale converges almost surely.
\end{corollary}

We need some ingredients before starting to tackle the proof. 

First we may assume that $(M_n)$ is a supermartingale.

The main idea is to introduce the notation of upcrossing. Fix $a < b$ and set 
$S_1 = \inf \{ n \geq 0 \colon M_n \leq a \} $, $T_1 = \inf \{ n \geq S_1 \colon
M_n \geq b\} $ and by induction $S_{k+1} = \inf \{ n \geq T_k \colon M_n \leq a
\}, T_{k+1} = \inf \{ n \geq S_{k+1} \colon M_n \geq b \} $ with the convention
$\inf \emptyset = \infty$. 

Then for $n \geq 1$, define $N_n([a, b]) = \sum_{k=1}^\infty \ind_{\{ T_k \leq n \} } $
which are the number of upcrossings of $[a, b]$ by $(M_n)_{n \geq 0}$ up to time
$n$.

\begin{lemma}
  $(M_n)_{n \geq 1}$ converges in $[-\infty, \infty]$ iff $\forall a < b, a, b
  \in \qq$, $N_{\infty}[a, b] < \infty$.
\end{lemma}

\begin{lemma}
  [Doob Upcrossing Lemma]
  Let $(M_n)$ be a supermartingale. Then $\forall a < b, \forall n \geq 1$
  $\ee[N_n([a, b])] \leq \frac{1}{b - a} \ee[(a - M_n)^+] $
\end{lemma}
\begin{proof}
    \hfill

    \noindent\textbf{\sffamily Step 1}
    Observe that $\forall k, n \geq 1$, $\{ T_k \leq n \}, \{ S_k \leq n \} \in
    \mathcal{F}_n$. The idea is to define $H_n = \sum_{k=1}^\infty \ind_{\{ S_k
    < n \leq T_k\} }$ which is one iff $M$ is in the process of doing an
    upcrossing at time $n$. Notice that this is predictable, as for each $k$,
    $\{ S_k < n \leq T_k \} = \{ S_k \leq n-1 \} \setminus \{ T_k \leq n-1 \}
    \in \mathcal{F}_{n-1}$.

    We now consider $(H \cdot M)_n$ which is a supermartingale. Write
    \begin{align*}
      (H \cdot M)_l &= 
          \sum_{n=1}^l H_n (M_n - M_{n-1}) \\
          &= \sum_{n=1}^l \sum_{k=1}^\infty \ind_{\{ S_k < n \leq T_k \} } (M_n -
          M_{n - 1}) \\
          &=
          \sum_{k=1}^\infty\sum_{n=1}^l  \ind_{\{ S_k < n \leq T_k \} } (M_n -
          M_{n - 1}) \\
          &= \sum_{k=1}^\infty \sum_{n=S_k + 1}^{\min(T_k, l)} (M_n - M_{n - 1})
          \\
          &= \sum_{k = 1}^{N_l([a, b])} (M_{T_k} - M_{S_k}) \\ 
          &+ \ind_{S_{N_l([a,
          b]) + 1}\leq l}(M_l - M_{S_{N_l([a, b]) + 1}}) \\
          &\geq (b - a) N_{l}([a, b]) - (a - M_l)^+
    \end{align*}

    But now notice that $(H \cdot M)_l$ is a supermartingale and $\ee[(H \cdot
    M)_0] = 0$, thus we get by taking expectation
    \[
      0 \geq \ee[(H \cdot M)_l] \geq (b-a) \ee[N_l([a, b])] - \ee[(a - M_l)^+]
    \]
    from which we get the result.
\end{proof}

\begin{proof}
  [Proof of the Theorem using the lemma]
  Take $(M_n)_{n \geq 0}$ a supermartingale, bounded in $L^1$. Set $K = \sup_{n
  \geq 1} \ee[|M_n|] < \infty$. By the "deterministic" upcrossing result, it is
  enough to show that $\forall a < b, a, b \in \qq$ almost surely
  $N_{\infty}([a, b])<\infty$. Indeed, we then have $a.s. \forall a < b, a, b
  \ni \qq N_{\infty}([a, b]) < \infty$ thus almost surely $(M_n)$ converges.

  First, by the Doob upcrossing lemma, $\ee[N_n([a, b])] \leq \frac{a + K}{b -a}
  $ but $N_n([a, b]) \to N_{\infty}([a, b])$ increasingly, thus by monotone
  convergence, 
  \[
    \ee[N_{\infty}([a, b])] = \lim_{n \to \infty} \ee[N_{n}([a, b])] \leq
    \frac{a + K}{b - a} < \infty
  .\] 
Thus $N_{\infty}([a, b]) < \infty$ almost surely. This shows that $M_n
  \overset{a.s.}{\longrightarrow} M_{\infty}$.

  Next we show that $\ee[|M_{\infty}|] < \infty$. By Fatou's Lemma:
  \[
    \ee[|M_{\infty}|] = \ee[\liminf_{n \to \infty} |M_n|] \leq \liminf_{n \to
    \infty}\ee[|M_n|] \leq K < \infty
  .\] 
\end{proof} 

\begin{remark}
    A (sub/super)martingale bounded in $L^p$ with $p > 1$ is also bounded in
    $L^1$ because $\ee[|X|] \leq \ee[|X|^p]^{1/p}$. If this is the case, it
    converges almost surely by Doob's theorem. 

    But we also seen that bounded in $L^p$ implies uniform integrability thus 
    $M_n \overset{\pp}{\longrightarrow} M_{\infty}$ and $(M_n)$ UI, so $M_n
    \overset{L^1}{\longrightarrow} M_{\infty}$.

    {\color{red} Warning} for this to hold $p$ must be strictly greater than
    $1$.
\end{remark}

\subsection{Example: The Bienaymé Galton-Watson branching processes}

\underline{Goal:} introducea simple model for the evolution of a population.

Let $\mu$ be a probability distribution on $\nn = \{ 0, 1, \ldots \} $.
Interpretation $\mu(k)$ is the probability of having $k$ children.

Let $(K_{n, j})_{n \geq 0, j \geq 1}$ be an iid family of $\mu-$ distributed
random variables. Define by induction $X_0 = 1$ and for $n \geq 0$ $X_{n+1} =
\sum_{j=1}^{X_n} K_{n, j}(w)$. Interpretation: $X_n$ is the size of the
population at generation $n$.

\vspace{0.5em}
\noindent
\textbf{\sffamily Question}
    What is the behavior of $X_n$ as $n \to \infty$?

\end{document}
