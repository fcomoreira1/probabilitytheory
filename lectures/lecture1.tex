\documentclass{scrartcl}

\author{Francisco Moreira Machado}

\title{Lecture 1}

\begin{document}
  % \maketitle
  \section{$\sigma-$fields and Measures}
  \subsection{$\sigma-$fields}
  \begin{definition*}
    Let $ \Omega $ be a set. A \underline{$\sigma-$field} $\mathcal{A}$ is a collection of
    subsets of $\Omega$ ($\mathcal{A}\subset\mathcal{P}(\Omega)$) such that.
    \begin{enumerate}
    \item $\Omega \in \mathcal{A}$.
    \item If $A \in \mathcal{A}$, $A^c \in \mathcal{A}$ (Stability by complement)
    \item If $(A_n)_{n \geq 1}$ is a sequence of elements of $\mathcal{A}$, then $\cup_{n \geq
      1} A_n $ (Stability by countable union).
    \end{enumerate}
  \end{definition*}

  $(\Omega, \mathcal{A})$ is called a measurable space. Elements of $\mathcal{A}$ are called
  measurable sets or events.

  \begin{example}
      Take a set $\Omega$,
      \begin{itemize}
        \item $A_1 = \{ \emptyset, \Omega \} $.
        \item $A_2 = \mathcal{P}(\Omega) $.
        \item $A_3 = \{ A \subset \Omega \colon A \text{ or } A^c \text{ are countable} \} $.
        \item $A_4 = \{ A \subset \nn \colon  A \text{ or } A^c \text{ are finite} \} $ is
          \underline{not} a $\sigma-$field.
        \marginnote{Exercise $\rightarrow$}[0cm]
      \end{itemize}
  \end{example}

  \begin{remark}
    [Trivial properties of $\sigma-$fields]
    \hfill
    \begin{itemize}
      \item We can easily derive from 1. and 2. that $\emptyset \in \mathcal{A}$.
      \item We can also derive from 2. and 3. that $\bigcap_{n \in \nn} A_n \in \mathcal{A}$.
    \end{itemize}
  \end{remark}

  Now to understand the intuition behind this definition, let us show a possible interpretation
  in Probability. $\Omega$ represents everything that can happen in a model, while elements in
  $\mathcal{A}$ are the sets an \textit{observer} is able to detect.

  \begin{definition}
    [Limsup and Liminf]
    Let  $(A_n)_{n \geq 1}$ be events of $(\Omega, \mathcal{A})$. We define
    \begin{itemize}
      \item $\limsup_{n \to \infty} A_n = \bigcap_{N\geq 0}\bigcup_{n \geq N} A_n$.
      \item $\liminf_{n \to \infty} A_n = \bigcup_{N\geq 0}\bigcap_{n \geq N} A_n$.
    \end{itemize}
  \end{definition}
  \begin{remark}
    For $\omega \in \Omega$ we have $\omega \in \limsup_{n \to \infty} A_n \iff \{ n \geq 1
    \colon \omega \in A_n \} $ is infinite. Moreover $\omega \in \liminf_{n \to \infty}A_n \iff
    \exists n(\omega) \text{ s.t. } n \geq n(\omega) \implies \omega \in A_n$.
    \marginnote{Exercise $\rightarrow$}[-1em]
  \end{remark}
  
  {\color{red} WARNING:} This should \underline{not} be confused with the usual notion of
  $\limsup$ and $\liminf$ for sequences of real numbers.

  \begin{proposition}
    Let $(A_i)_{i \in I}$ be a collection of $\sigma-$fields on $\Omega$ ($I$ not necessarily
    countable). Then, $\bigcap_{i \in I} A_i$ is itself a $\sigma-$field.
  \end{proposition}
  \begin{proof}
      It suffices to check the three properties of $\sigma-$fields.
      \begin{enumerate}
        \item $\Omega \in \mathcal{A}_i$ $\forall i \in I$, thus it is in $\bigcap_{i \in I}
          \mathcal{A}_i$.

        \item If $A \bigcap_{i \in I} \mathcal{A}_i$, then $A \in \mathcal{A}_i \;\forall i \in
          I$, hence $A^c \in \mathcal{A}_i$ $\forall i \in I$, hence $A^c \in \bigcap_{i \in I}
          \mathcal{A}_i$.

        \item Similar reasoning \marginnote{Exercise $\rightarrow$}
      \end{enumerate}
  \end{proof}

  \subsubsection{Generated $\sigma-$field}
  \begin{definition}
    If $\mathcal{C} \subset \mathcal{P}(\Omega)$ is a collection of subsets of $\Omega$. We
    define 
    $$\sigma(\mathcal{C}) = \bigcap_{\substack{\mathcal{A} \text{ is a } \sigma-\text{field} \\
    \mathcal{C} \subset \mathcal{A}}} \mathcal{A}$$
    which is called the $\sigma-$field generated by $\mathcal{C}$. 
  \end{definition}

  Notice that the generated $\sigma-$field by $\mathcal{C}$ is indeed a $\sigma-$field by
  proposition 1.2. Moreover, this is an intersection of at least one element, as
  $\mathcal{P}(\mathcal{C})$ satisfies the conditions.

  Finally, this is the \textbf{smallest} $\sigma-$field containing $\mathcal{C}$. This
  construction is particularly useful as it is hard to explicitly construct such a field due to
  the possible uncountability.

  \begin{remark}
      If $\mathcal{C}$ is a $\sigma-$field, then $\sigma(\mathcal{C}) = \mathcal{C}$.
  \end{remark}

  \begin{proposition}
      If $\mathcal{C} \subset \mathcal{C}'$ then $\sigma(\mathcal{C}) \subset
      \sigma(\mathcal{C}')$. 
  \end{proposition}
  
\end{document}
